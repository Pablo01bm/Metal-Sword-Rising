\documentclass[a4paper,11pt]{book}
%\documentclass[a4paper,twoside,11pt,titlepage]{book}
\usepackage{listings}
\usepackage[utf8]{inputenc}
\usepackage[spanish]{babel}
\usepackage{graphicx}
\usepackage{caption}
\usepackage[font=it,labelfont=bf]{caption}
\usepackage{natbib}
\usepackage{float}
\usepackage{subcaption}



\graphicspath{ {./images/} }

% \usepackage[style=list, number=none]{glossary} %
%\usepackage{titlesec}
%\usepackage{pailatino}

\decimalpoint
\usepackage{dcolumn}
\newcolumntype{.}{D{.}{\esperiod}{-1}}
\makeatletter
\addto\shorthandsspanish{\let\esperiod\es@period@code}
\makeatother


%\usepackage[chapter]{algorithm}
\RequirePackage{verbatim}
%\RequirePackage[Glenn]{fncychap}
\usepackage{fancyhdr}
\usepackage{graphicx}
\usepackage{afterpage}

\usepackage{longtable}

\usepackage[pdfborder={000}]{hyperref} %referencia

% ********************************************************************
% Re-usable information
% ********************************************************************
\newcommand{\myTitle}{Título del proyecto\xspace}
\newcommand{\myDegree}{Grado en ...\xspace}
\newcommand{\myName}{Nombre Apllido1 Apellido2 (alumno)\xspace}
\newcommand{\myProf}{Nombre Apllido1 Apellido2 (tutor1)\xspace}
\newcommand{\myOtherProf}{Nombre Apllido1 Apellido2 (tutor2)\xspace}
%\newcommand{\mySupervisor}{Put name here\xspace}
\newcommand{\myFaculty}{Escuela Técnica Superior de Ingenierías Informática y de
Telecomunicación\xspace}
\newcommand{\myFacultyShort}{E.T.S. de Ingenierías Informática y de
Telecomunicación\xspace}
\newcommand{\myDepartment}{Departamento de ...\xspace}
\newcommand{\myUni}{\protect{Universidad de Granada}\xspace}
\newcommand{\myLocation}{Granada\xspace}
\newcommand{\myTime}{\today\xspace}
\newcommand{\myVersion}{Version 0.1\xspace}


\hypersetup{
pdfauthor = {\myName (email (en) ugr (punto) es)},
pdftitle = {\myTitle},
pdfsubject = {},
pdfkeywords = {palabra_clave1, palabra_clave2, palabra_clave3, ...},
pdfcreator = {LaTeX con el paquete ....},
pdfproducer = {pdflatex}
}

%\hyphenation{}


%\usepackage{doxygen/doxygen}
%\usepackage{pdfpages}
\usepackage{url}
\usepackage{colortbl,longtable}
\usepackage[stable]{footmisc}
%\usepackage{index}

\makeindex
\usepackage[style=long, cols=2,border=plain,toc=true,number=none]{glossary}
\makeglossary

% Definición de comandos que me son tiles:
%\renewcommand{\indexname}{Índice alfabético}
%\renewcommand{\glossaryname}{Glosario}

\pagestyle{fancy}
\fancyhf{}
\fancyhead[LO]{\leftmark}
\fancyhead[RE]{\rightmark}
\fancyhead[RO,LE]{\textbf{\thepage}}
\renewcommand{\chaptermark}[1]{\markboth{\textbf{#1}}{}}
\renewcommand{\sectionmark}[1]{\markright{\textbf{\thesection. #1}}}

\setlength{\headheight}{1.5\headheight}

\newcommand{\HRule}{\rule{\linewidth}{0.5mm}}
%Definimos los tipos teorema, ejemplo y definición podremos usar estos tipos
%simplemente poniendo \begin{teorema} \end{teorema} ...
\newtheorem{teorema}{Teorema}[chapter]
\newtheorem{ejemplo}{Ejemplo}[chapter]
\newtheorem{definicion}{Definición}[chapter]

\definecolor{gray97}{gray}{.97}
\definecolor{gray75}{gray}{.75}
\definecolor{gray45}{gray}{.45}
\definecolor{gray30}{gray}{.94}

\lstset{ frame=Ltb,
     framerule=0.5pt,
     aboveskip=0.5cm,
     framextopmargin=3pt,
     framexbottommargin=3pt,
     framexleftmargin=0.1cm,
     framesep=0pt,
     rulesep=.4pt,
     backgroundcolor=\color{gray97},
     rulesepcolor=\color{black},
     %
     stringstyle=\ttfamily,
     showstringspaces = false,
     basicstyle=\scriptsize\ttfamily,
     commentstyle=\color{gray45},
     keywordstyle=\bfseries,
     %
     numbers=left,
     numbersep=6pt,
     numberstyle=\tiny,
     numberfirstline = false,
     breaklines=true,
   }
 
% minimizar fragmentado de listados
\lstnewenvironment{listing}[1][]
   {\lstset{#1}\pagebreak[0]}{\pagebreak[0]}

\lstdefinestyle{CodigoC}
   {
	basicstyle=\scriptsize,
	frame=single,
	language=C,
	numbers=left
   }
\lstdefinestyle{CodigoC++}
   {
	basicstyle=\small,
	frame=single,
	backgroundcolor=\color{gray30},
	language=C++,
	numbers=left
   }

 
\lstdefinestyle{Consola}
   {basicstyle=\scriptsize\bf\ttfamily,
    backgroundcolor=\color{gray30},
    frame=single,
    numbers=none
   }

   \lstdefinestyle{customcsharp}{
    language=[Sharp]C,
    basicstyle=\ttfamily\small,
    keywordstyle=\color{blue},
    commentstyle=\color{green!60!black},
    stringstyle=\color{orange},
    showstringspaces=false,
    tabsize=4,
    breaklines=true,
    breakatwhitespace=true,
    numbers=left,
    numberstyle=\tiny\color{gray},
    frame=single,
    captionpos=b
}


\newcommand{\bigrule}{\titlerule[0.5mm]}


%Para conseguir que en las páginas en blanco no ponga cabecerass
\makeatletter
\def\clearpage{%
  \ifvmode
    \ifnum \@dbltopnum =\m@ne
      \ifdim \pagetotal <\topskip
        \hbox{}
      \fi
    \fi
  \fi
  \newpage
  \thispagestyle{empty}
  \write\m@ne{}
  \vbox{}
  \penalty -\@Mi
}
\makeatother

\usepackage{pdfpages}
\begin{document}
\begin{titlepage}
 
 
\newlength{\centeroffset}
\setlength{\centeroffset}{-0.5\oddsidemargin}
\addtolength{\centeroffset}{0.5\evensidemargin}
\thispagestyle{empty}

\noindent\hspace*{\centeroffset}\begin{minipage}{\textwidth}

\centering
\includegraphics[width=0.9\textwidth]{imagenes/logo_ugr.jpg}\\[1.4cm]

\textsc{ \Large TRABAJO FIN DE GRADO\\[0.2cm]}
\textsc{ INGENIERÍA INFORMÁTICA}\\[1cm]
% Upper part of the page
% 
% Title
{\Huge\bfseries Desarrollo de videojuegos\\
}
\noindent\rule[-1ex]{\textwidth}{3pt}\\[3.5ex]
{\large\bfseries Generación procedural, modificación de geometría y sombreados }
\end{minipage}

\vspace{2.5cm}
\noindent\hspace*{\centeroffset}\begin{minipage}{\textwidth}
\centering

\textbf{Autor}\\ {Pablo Borrego Megías}\\[2.5ex]
\textbf{Directores}\\
{Carlos Ureña Almagro}\\[2cm]
\includegraphics[width=0.3\textwidth]{imagenes/etsiit_logo.png}\\[0.1cm]
\textsc{Escuela Técnica Superior de Ingenierías Informática y de Telecomunicación}\\
\textsc{---}\\
Granada, Junio de 2023
\end{minipage}
%\addtolength{\textwidth}{\centeroffset}
%\vspace{\stretch{2}}
\end{titlepage}



% \chapter*{}
% %\thispagestyle{empty}
% %\cleardoublepage

% %\thispagestyle{empty}

% %\begin{titlepage}
 
 
\setlength{\centeroffset}{-0.5\oddsidemargin}
\addtolength{\centeroffset}{0.5\evensidemargin}
\thispagestyle{empty}

\noindent\hspace*{\centeroffset}\begin{minipage}{\textwidth}

\centering
%\includegraphics[width=0.9\textwidth]{imagenes/logo_ugr.jpg}\\[1.4cm]

%\textsc{ \Large PROYECTO FIN DE CARRERA\\[0.2cm]}
%\textsc{ INGENIERÍA EN INFORMÁTICA}\\[1cm]
% Upper part of the page
% 

 \vspace{3.3cm}

%si el proyecto tiene logo poner aquí
\includegraphics{imagenes/logo.png} 
 \vspace{0.5cm}

% Title

{\Huge\bfseries Título del proyecto\\
}
\noindent\rule[-1ex]{\textwidth}{3pt}\\[3.5ex]
{\large\bfseries Subtítulo del proyecto.\\[4cm]}
\end{minipage}

\vspace{2.5cm}
\noindent\hspace*{\centeroffset}\begin{minipage}{\textwidth}
\centering

\textbf{Autor}\\ {Nombre Apellido1 Apellido2 (alumno)}\\[2.5ex]
\textbf{Directores}\\
{Nombre Apellido1 Apellido2 (tutor1)\\
Nombre Apellido1 Apellido2 (tutor2)}\\[2cm]
%\includegraphics[width=0.15\textwidth]{imagenes/tstc.png}\\[0.1cm]
%\textsc{Departamento de Teoría de la Señal, Telemática y Comunicaciones}\\
%\textsc{---}\\
%Granada, mes de 201
\end{minipage}
%\addtolength{\textwidth}{\centeroffset}
\vspace{\stretch{2}}

 
\end{titlepage}






% \cleardoublepage
% \thispagestyle{empty}

% \begin{center}
% {\large\bfseries Título del Proyecto: Subtítulo del proyecto}\\
% \end{center}
% \begin{center}
% Pablo Borrego Megías \\
% \end{center}

% %\vspace{0.7cm}
% \noindent{\textbf{Palabras clave}: palabra\_clave1, palabra\_clave2, palabra\_clave3, ......}\\

% \vspace{0.7cm}
% \noindent{\textbf{Resumen}}\\

% Poner aquí el resumen.
% \cleardoublepage


% \thispagestyle{empty}


% \begin{center}
% {\large\bfseries Project Title: Project Subtitle}\\
% \end{center}
% \begin{center}
% First name, Family name (student)\\
% \end{center}

% %\vspace{0.7cm}
% \noindent{\textbf{Keywords}: Keyword1, Keyword2, Keyword3, ....}\\

% \vspace{0.7cm}
% \noindent{\textbf{Abstract}}\\

% Write here the abstract in English.

\chapter*{}
\thispagestyle{empty}

\noindent\rule[-1ex]{\textwidth}{2pt}\\[4.5ex]

Yo, \textbf{Pablo Borrego Megías}, alumno de la titulación grado en Ingeniería Informática de la \textbf{Escuela Técnica Superior
de Ingenierías Informática y de Telecomunicación de la Universidad de Granada}, con DNI 26504976Y, autorizo la
ubicación de la siguiente copia de mi Trabajo Fin de Grado en la biblioteca del centro para que pueda ser
consultada por las personas que lo deseen.

\vspace{6cm}

\noindent Fdo: Pablo Borrego Megías

\vspace{2cm}

\begin{flushright}
Granada a 27 de Junio de 2023 .
\end{flushright}


\chapter*{}
\thispagestyle{empty}

\noindent\rule[-1ex]{\textwidth}{2pt}\\[4.5ex]

D. \textbf{Carlos Ureña Almagro}, Profesor del Departamento de Lenguajes y Sistemas Informáticos de la Universidad de Granada.

\vspace{0.5cm}

\textbf{Informa:}

\vspace{0.5cm}

Que el presente trabajo, titulado \textit{\textbf{Desarrollo de videojuegos, Generación procedural, modificación de geometría y sombreados}},
ha sido realizado bajo su supervisión por \textbf{Pablo Borrego Megías}, y autorizo la defensa de dicho trabajo ante el tribunal
que corresponda.

\vspace{0.5cm}

Y para que conste, expide y firma el presente informe en Granada a 27 de Junio de 2023 .

\vspace{1cm}

\textbf{El director:}

\vspace{5cm}

\noindent \textbf{Carlos Ureña Almagro}

\chapter*{Agradecimientos}
\thispagestyle{empty}

       \vspace{1cm}


Muchas gracias a mis amigos y familiares, por ser un pilar fundamental en mi vida y estar siempre apoyándome en cualquier momento.


\frontmatter
\tableofcontents
\listoffigures
% \listoftables

\mainmatter
\setlength{\parskip}{5pt}

\chapter{Resumen}

\begin{center}
    {\large\bfseries Título del Proyecto: Subtítulo del proyecto}\\
    \end{center}
    \begin{center}
    Pablo Borrego Megías \\
    \end{center}

\chapter{Anstract}

\begin{center}
    {\large\bfseries Título del Proyecto: Subtítulo del proyecto}\\
    \end{center}
    \begin{center}
    Pablo Borrego Megías \\
    \end{center}

\chapter{Introducción y motivación}

Este proyecto surge como consecuencia de una idea ocurrida mientras se jugó el videojuego \textit{Metal Gear Rising} el cual incluye una mecánica que consiste en dar la posibilidad al jugador de cortar de forma precisa elementos del entorno, teniendo control en la dirección del corte. Tras observar dicha mecánica, se pensó en cómo se habría podido desarrollar tal efecto en un entorno 3D con la precisión y el realismo con el que está implementado. Esto, sumado a la idea de desarrollar un videojuego desde hace muchos años, ha resultado en la ocurrencia de desarrollar un juego implementando en él mecánicas realistas y generando los mapas de forma aleatoria, ya que para el diseño de los mismo es necesario tener unos conocimientos de diseño 3D previos. También debido a la reciente tendencia en videojuegos famosos como \textit{The Legend of Zelda} de realizar un estilo gráfico \textit{Toon Shading} en los videojuegos, alejándolos del típico efecto fotorealista, se decidió también recrear dicho estilo para el proyecto. 

\section{Contexto}

\subsection{Concepto e historia de la animación}

Antes de tratar los videojuegos en sí, es recomendable primero introducir un aspecto importante de ellos, que es la animación, ya que es un concepto del que han derivado desde su origen elementos como los videojuegos, el cine de animación, series y demás como los conocemos hoy día. La animación es el proceso de crear la ilusión de movimiento a partir de una secuencia de imágenes estáticas. Se logra mostrando una serie de imágenes en rápida sucesión, cada una ligeramente diferente de la anterior, lo que crea la ilusión de movimiento continuo. Pueden ser dibujos, imágenes generadas por computadora o incluso objetos reales que se mueven cuadro a cuadro.

La animación \cite{Animacion} tal y como la conocemos hoy día comenzó cuando los animadores de \textbf{Disney}, \textbf{Ollie Johnston} y \textbf{Frank Thomas}, recogieron en su libro \textit{The Illusion of life} los "Doce principios básicos de la animación" \cite{TheIllusionOfLife}. El objetivo de estos principios era intentar crear la ilusión de que los personajes se apegaban a las leyes de la física aunque se abarcaron también temas como el tiempo emocional y atractivo de los personajes. 

\begin{figure}[H]
   \centering
   \includegraphics[width=8cm, height=8cm]{pelotaRebotando.png}
   \caption{Bola roja rebotando desglosada en 6 fotogramas. Extraido de https://es.wikipedia.org/wiki/Archivo:Animexample3edit.png }
\end{figure}

Aunque inicialmente se pretendía que estos principios se aplicaran principalmente a la animación tradicional o animación dibujada a mano, siguen siendo de gran relevancia en la actualidad, especialmente en el contexto de la animación por ordenador que prevalece hoy en día.


\subsection{Origen de los videojuegos}

Antes de poder tratar el proyecto que se ha realizado, conviene repasar la historia de los videojuegos, los tipos que hay y los elementos que los componen.

Los videojuegos han evolucionado de manera significativa desde sus primeros días hasta convertirse en una industria multimillonaria y una forma de entretenimiento globalmente popular, para ello se recorrerá brevemente las décadas desde su origen hasta los tiempos actuales.

\subsubsection{Evolución de los videojuegos}

\begin{enumerate}
    \item \textbf{Décadas de 1950 y 1960} Los primeros desarrollos en los videojuegos tuvieron lugar en laboratorios de investigación y universidades durante la década de 1950. El primer juego interactivo, llamado \textit{Tennis for Two}, fue creado por \textbf{William Higinbotham} en 1958 y se jugaba en un osciloscopio. A medida que avanzaba la década de 1960, los videojuegos se convirtieron en una curiosidad científica, con la creación de juegos como \textit{Spacewar!} en 1962, por lo que en ese momento se empezó a valorar el potencial que podían tener este tipo de productos.
    \item \textbf{Década de 1970} La década de 1970 marcó el comienzo de la industria moderna de los videojuegos tal y como la conocemos actualmente. El lanzamiento de la primera consola de videojuegos doméstica, la \textit{Magnavox Odyssey}, en 1972, abrió el camino para la popularización de los videojuegos. Sin embargo, fue el lanzamiento del juego \textit{Pong} de la famosa empresa tecnológica \textbf{Atari} en 1972 el que realmente atrajo la atención del público y estableció los cimientos de la industria. La mayoría de personas que vivieron esa época están de acuerdo en que el \textit{Pong} fue de los primeros videojuegos que probaron.
    \begin{figure}[H]
        \centering
        \includegraphics[width=9cm, height=6cm]{Pong.png}
        \caption{Atari Pong, extraído de \url{https://es.wikipedia.org/wiki/Pong}}
    \end{figure}
    \item \textbf{Década de 1980} La década de 1980 fue una época de rápido crecimiento y evolución para los videojuegos. El surgimiento de las consolas domésticas, como el \textit{Atari 2600} y el \textit{Nintendo Entertainment System (NES)}, permitió que los videojuegos llegaran a un público más amplio llevandolos a las casas de las personas particulares y enfocándolos a públicos más jóvenes. Juegos icónicos como \textit{Super Mario Bros} y \textit{Tetris} se convirtieron en éxitos mundiales y sentaron las bases para muchos géneros y mecánicas de juego que se utilizan hasta hoy.
    \item \textbf{Década de 1990} La década de 1990 fue testigo de un gran avance en los gráficos y la jugabilidad de los videojuegos. La introducción de consolas como la \textit{Super Nintendo Entertainment System} y la \textit{Sega Genesis} trajo consigo avances tecnológicos significativos sobre todo en el apartado gráfico de éstos, ya que se empezó a popularizar el 3D. Además, el auge de los juegos de PC permitió una mayor diversidad y complejidad en los juegos. Títulos influyentes como \textit{Doom}, \textit{Final Fantasy VII} y \textit{Super Mario 64} dejaron una marca duradera en la industria.
    \item \textbf{Década de 2000} El comienzo del siglo XXI trajo consigo una nueva era para los videojuegos. El avance en la tecnología de gráficos 3D, junto con el aumento de la conectividad en línea (juegos on-line), abrió la puerta a una mayor interacción entre los jugadores. Los juegos en línea y los juegos multijugador masivos en línea (MMO) se volvieron extremadamente populares. Ésta época marcó la tendencia de enfocar cada vez más los videojuegos el juego con otros mediante internet. 
    \item \textbf{Década de 2010 en adelante} La última década ha sido testigo de una explosión en la popularidad de los videojuegos y su integración en la cultura popular, tanto que gente de todas las edades, a día de hoy es fácil que jueguen a algún videojuego. Los avances tecnológicos en las consolas de última generación, como \textit{PlayStation 4},\textit{Xbox One} y \textit{Nintendo Switch}, han llevado los gráficos y la jugabilidad a niveles sin precedentes. Además, la aparición de los juegos móviles y las plataformas de transmisión en vivo, como \textit{Twitch}, han transformado la forma en que se juegan y se consumen los videojuegos como algo disfrutable en comunidad.
\end{enumerate}

\subsubsection{Actualidad de los videojuegos}

En la actualidad, los videojuegos se han convertido en un fenómeno global, con millones de jugadores en todo el mundo. La realidad virtual, la realidad aumentada y la inteligencia artificial están abriendo nuevas posibilidades en el diseño y la experiencia de juego. Los videojuegos también han encontrado su lugar en los deportes electrónicos (eSports), con torneos y competiciones que atraen a grandes audiencias y ofrecen premios millonarios.

\subsection{Evolución de la capacidad gráfica de los ordenadores}

A medida que las unidades de procesamiento gráfico (GPUs) se volvieron más comunes en los dispositivos informáticos, se produjo un gran avance en el rendimiento y las capacidades gráficas.

En los años 90, las GPUs empezaron a utilizarse principalmente en ordenadores de sobremesa y consolas de videojuegos. En ese momento, las tarjetas gráficas tenían un rendimiento limitado en comparación con las versiones más actuales que podemos encontrar, pero en ese entonces, su capacidad para acelerar los cálculos gráficos era revolucionaria. Estas primeras GPUs se centraban en procesar polígonos y texturas en entornos tridimensionales, lo que permitía la creación de gráficos más realistas y detallados.

A medida que avanzaba la década de 1990, las GPUs mejoraron rápidamente sus capacidades. Se añadieron nuevas características, como por ejemplo el mapeado de texturas, la iluminación y los efectos especiales, lo que produjo un aumento de la calidad visual de los juegos y aplicaciones gráficas. Estas mejoras permitieron a los desarrolladores crear mundos virtuales más inmersivos y realistas.

En los primeros años de la década de los 2000, las tarjetas gráficas empezaron a integrarse en los ordenadores portátiles, lo que permitió a los usuarios disponer de gráficos avanzados en dispositivos más compactos. Las GPUs móviles se volvieron más eficientes y potentes, lo que resultó en una experiencia gráfica similar a la de los ordenadores de sobremesa.

Con el tiempo, las gráficas evolucionaron aún más y se hicieron aún más potentes. Se introdujeron nuevas tecnologías, como los sombreadores programables (\textit{shaders}), que permitían un mayor control sobre los efectos visuales y una mayor flexibilidad en la representación de gráficos. Esto dio lugar a mejoras significativas en la calidad visual de los videojuegos, películas y otras aplicaciones gráficas.

A medida que los dispositivos móviles se hicieron más populares, las tarjetas de gráficos también se integraron en este tipo de dispositivos. Los \textit{smartphones} y las tabletas comenzaron a aprovechar las GPUs para ofrecer gráficos mejorados en pantallas más pequeñas. Esto permitió la reproducción de juegos de alta calidad, así como aplicaciones de realidad aumentada y realidad virtual en dispositivos móviles.

Además, las GPUs han tenido un impacto significativo en el campo del aprendizaje automático y la \textbf{inteligencia artificial}. Gracias a su capacidad para realizar cálculos paralelos masivos, las tarjetas gráficas se han convertido en una herramienta esencial en el procesamiento de datos para aplicaciones IA, como el reconocimiento de imágenes y el procesamiento del lenguaje natural.

La popularización del uso de las GPUs a partir de los años 90 ha impulsado una notable evolución en las capacidades gráficas de los ordenadores y dispositivos. Desde la mejora de la calidad visual en juegos y aplicaciones gráficas, hasta la integración de GPUs en dispositivos portátiles y móviles, estas unidades de procesamiento gráfico han desempeñado un papel fundamental en el desarrollo de experiencias visuales más inmersivas y avanzadas en una amplia gama de dispositivos.

\section{Componentes de los videojuegos}

Los videojuegos se componen de una serie de elementos que, en conjunto, crean una experiencia única y envolvente para los jugadores. Estos componentes han evolucionado a lo largo del tiempo y han contribuido a que los videojuegos sean considerados una forma de arte para muchos usuarios. A continuación se tratará de forma breve cada uno de estos componentes.

\textbf{Gráficos y Diseño Visual}. Los avances en la tecnología han permitido la creación de gráficos 3D cada vez más realistas en los videojuegos. Los programadores usan motores gráficos y herramientas de diseño para crear esos mundos. Por otra parte, el desarrollo de gráficos de buena calidad suele repercutir en el coste y cantidad de recursos necesarios para mover el juego con fluidez, lo que puede plantear desafíos técnicos y de rendimiento.

\textbf{Jugabilidad y Mecánicas de Juego}. La jugabilidad es uno de los aspectos más importantes de un videojuego. Las mecánicas de juego, como los controles, las físicas y las interacciones del jugador con el entorno, determinan cómo será la experiencia para el jugador dentro del entorno. Los diseñadores deben equilibrar la dificultad, la accesibilidad y la diversión para garantizar una buena experiencia a los usuarios. El diseño de niveles es una parte fundamental de la jugabilidad.

\textbf{Historia} Muchos videojuegos cuentan con historias complejas. La escritura de guiones y diálogos, así como la creación de personajes memorables, son componentes cruciales para involucrar emocionalmente a los jugadores en el videojuego. Sin embargo, la creación de una narrativa puede ser un desafío, debido a que se debe equilibrar con la jugabilidad y mantener el ritmo adecuado para mantener el interés del jugador. Demasiadas líneas de diálogo o fragmentos cinematográficos pueden desagradar a los usuarios.

\textbf{Música y sonido}. La música y los efectos de sonido tienen un papel vital en la creación de emociones para el jugador dentro de un videojuego. La composición de bandas sonoras originales y la implementación de sonidos requieren un nivel de conocimiento de música y composición que en la mayoria de los casos hace necesaria la contratación de alguien especializado en estos aspectos. Además, los problemas de licencia y derechos de autor pueden surgir al utilizar música existente.

\textbf{Personajes y arte conceptual}. El diseño de personajes y el arte conceptual son elementos esenciales para la identidad de un videojuego. Los artistas digitales crean ilustraciones, modelos 3D, animaciones y efectos visuales que dan vida a los personajes y al mundo del juego.

Gracias a la combinación de estos componentes, los videojuegos se han llegado a considerar un nuevo tipo de arte. A través de la creatividad, la tecnología y la innovación, los desarrolladores han logrado crear experiencias interactivas que provocan emociones a los jugadores y en ocasiones buscan transmitir mensajes.

Llegado a este punto, da la sensación de que para poder desarrollar un videojuego, son necesarios los conocimientos en múltiples campos y la disposición de gran cantidad de herramientas para ello, y esto, para desarrolladores independientes suele ser algo totalmente imposible, por lo que es muy frustante esa falta de medios o de conocimientos. 

\subsubsection{Valoración de distintos motores gráficos}

Con el desarrollo de herramientas como \textit{Unity}, se ha conseguido dar la posibilidad a los desarrolladores independientes de usar una herramienta software con todos los medios necesarios para el desarrollo de un videojuego. \textit{Unity}  es uno de los motores de videojuegos más populares del mercado. Según estadísticas recientes, se estima que más del 60\verb|%| de todos los juegos móviles y alrededor del 50\verb|%| de los juegos para PC y consolas se crean utilizando \textit{Unity}. Esto se debe a su accesibilidad tanto para desarrolladores independientes como para grandes estudios. También es compatible con una amplia gama de plataformas por lo que es una buena opción para la mayoría de los desarrolladores. También tiene soporte para realidad virtual. Permite además desarrollar proyectos tanto en 2D como 3D por lo que se ofrece una gran posibilidad a los desarrolladores para que realicen sus ideas en este motor. Un gran punto a favor de Unity es su extensa documentación y la gran comunidad que tiene, ya que si surge algún problema, en la documentación se puede encontrar la solución, o bien en algún foro se ha resuelto una duda parecida, por lo que resulta muy útil.

Unity ofrece numerosas características avanzadas que permiten a los desarrolladores crear experiencias de juego más inmersivas. Algunas de estas características incluyen herramientas de física, sistemas de partículas, renderizado de alta calidad, inteligencia artificial, redes y multijugador.

Otras herramientas que desempeñan un papel similar al de \textit{Unity} son \textit{Unreal Engine} y \textit{Godot}.

Unreal Engine es un motor de gráficos utilizado principalmente en el desarrollo de videojuegos. Utiliza C++ como lenguaje de programación, lo que permite un control preciso y un alto rendimiento. Destaca por sus gráficos avanzados, ofreciendo un motor de renderizado realista y efectos visuales de alta calidad. Además, cuenta con un sistema de programación visual llamado \textit{Blueprints}, que facilita la creación de lógica de juego sin necesidad de programar en código. También incluye un sistema de simulación física para interacciones realistas. Unreal Engine cuenta con una gran comunidad de desarrolladores, amplia documentación, tutoriales y soporte técnico, lo que lo convierte en una buena opción, poderosa y versátil.

Por otro lado, Godot es otro motor de juego utilizado en el desarrollo de videojuegos. Utiliza \textit{GDScript}, un lenguaje basado en \textit{Python}, que es fácil de aprender y utilizar. Godot tiene un enfoque en el sistema de nodos, permitiendo construir la estructura y la lógica del juego de forma intuitiva. Además, ofrece un editor unificado que combina el diseño de niveles y la programación en un solo entorno. Es compatible con múltiples plataformas, lo que permite exportar juegos a diferentes sistemas operativos de forma sencilla. Una ventaja adicional de Godot es que es de \textbf{código abierto}, lo que significa que es gratuito y se puede personalizar según las necesidades del desarrollador. Es una opción popular para desarrolladores independientes y equipos pequeños debido a su facilidad de uso y su licencia de código abierto.

Tras valorar las opciones mencionadas, se optó finalmente por usar \textit{Unity} por las siguientes razones, su comunidad es la más grande actualmente en cuanto a usuarios, documentación y foros, por lo que para encontrar ayuda a la hora de desarrollar sería más sencillo incluso que \textit{Unreal Engine}. El lenguaje que se usa es \textit{C\#}, el cual se tiene un gran interés en aprender. Para desarrollar videojuegos 3D es mejor opción que \textit{Godot} y entre \textit{Unity} y \textit{Unreal Engine} se eligió \textit{Unity} debido a que se pueden crear gráficos no realistas con más facilidad que en \textit{Unreal Engine}, el cual está enfocado en gráficos realistas de alta calidad. Finalmente, la razón principal fue que en el momento de elegir, se consiguió acceso a cursos de aprendizaje de \textit{Unity} de pago por buen precio y muy completos, por lo que tras realizarlos finalmente se optó por dicha herramienta.

Por éstas razones finalmente se eligió \textit{Unity} para el desarrollo del proyecto.

\section{Descripción del problema}

El objetivo del trabajo será el desarrollo de un videojuego interactivo el cual pueda resultar en una buena experiencia en el usuario final. Para ello será necesario enfrentarse a una serie de problemas para poder obtener el resultado estimado.

Será clave definir el tipo de juego que se quiere, así como el tipo de cámara que se usará para mostrar los eventos y desarrollar alrededor de ello el resto de videojuego. La temática artística será también otro tema a tener en cuenta.

Se prosigue con la definición del personaje que controlará el jugador, se ha de buscar un modelo que satisfaga el estilo definido previamente, y dotar a dicho personaje de elementos como un arma para enfrentarse a los enemigos y un set de movimientos acompañados con sus respectivas animaciones.

Otro aspecto que se tendrá que resolver es el de la modificación de las geometrías de las escena, ya que al jugador se le dotará de la posibilidad de realizar ''Cortes'' a los elementos estáticos de la escena, y se deben ver modificados de forma fluida para que la experiencia del jugador sea adecuada. También dar la posibilidad de modificar la dirección del corte.

La generación de los mapas y sus elementos de forma procedural será otro apartado que tener en cuenta, ya que para evitar el diseño de un mapa completo y estático, se ha decidido que se genere de forma procedural, que eso conlleva explorar diferentes posibilidades y aplicar conocimientos adquiridos en asignaturas como \textit{Informática Gráfica} y \textit{Sistemas Gráficos}.

Para desarrollar el estilo gráfico deseado \textit{Toon Shading} será necesario investigar el comportamiento de la luz sobre las caras que forman las mallas de los objetos y modificar ese comportamiento por defecto para conseguir el estilo deseado. Se requerirá el aprendizaje de herramientas propias de \textit{Unity} para facilitar el proceso de desarrollo.

Para conseguir superar la mayoría de estos problemas será necesario aprender a usar por tanto, el entorno de \textit{Unity} y el lenguaje de programación usado para el desarrollo de los \textit{Scripts} de código, necesarios para conseguir los comportamientos deseados. El lenguaje de programación es \textbf{C\#}, el cual es un lenguaje desarrollado por \textit{Microsoft} como parte de la plataforma \textbf{.NET}. La sintáxis está creada a partir de \textbf{C/C++} y su modelo de datos deriva de la citada plataforma, que es parecido al de \textit{Java}. Como se ha comentado, será el lenguaje usado para desarrollar los ficheros de comportamientos específicos de \textit{Unity}, para conseguir los resultados esperados.

\section{Alcance de la memoria}

La memoria sigue una estructura la cual abarca diferentes aspectos del desarrollo del proyecto realizado. En la siguiente sección, el capítulo \ref{chap:objetivos} se comentan los objetivos iniciales y los que se han conseguido finalmente. Después en el capípulo \ref{chap:resolucion} se detallará el desarrollo del programa en una serie de sub apartados. Se comenzará con la planificación inicial y el presupuesto para desarrollar el videojuego en la sección \ref{sec:planificacion}. 

Posteriormente en la sección \ref{sec:analisis} se realizará el análisis de requisitos e historias de usuario. A continuación, en la sección \ref{sec:implementacion} se tratará la parte más técnica del documento, mostrando bocetos del videojuego, diagramas de clases, explicación de los algoritmos usados y la implementación del videojuego, incluyendo imágenes de muestra de los resultados obtenidos así de ejemplos de código desarrollado.

En la sección \ref{sec:manual} se expone el manual de usuario, donde se indicará como poder usar el sofware así como la instalación del mismo.

En el capítulo \ref{chap:conclusiones} se expondrán las conclusiones del proyecto así como sus posibles vías futuras.

Por último se encuentra la bibliografía, dónde estará disponibles las fuentes consultadas para la realización de éste proyecto.

\chapter{Objetivos}
\label{chap:objetivos}

Desde el comienzo, se había tenido en mente que el objetivo general del proyecto sería un videojuego pero debido a la gran cantidad de opciones que implica ésto no se tenía muy claro cuales iban a ser las características que iba a tener. Sin embargo tras jugar a algunos títulos diferentes de videojuegos se decidió realizar una mezcla de  mecánicas vistas en juegos diferentes pero en uno sólo. Además la curiosidad de cómo es posible recrear en entornos 3D dichas mecánicas vistas hizo que se decidieran finalmente los distintos objetivos que iba a tener el proyecto. Por lo que el objetivo general pasó a ser la realización de un video con generación procedural de geometría y modificación de ésta, así como la implementación de un cauce gráfico para conseguir el estilo visual deseado.

Para conseguir realizar el objetivo principal, era necesario adquirir una base de conocimientos y experiencia en la herramienta con la que se iba a desarrollar, en éste caso es el motor de videojuegos \textit{Unity} así como el estudio previo de si dicha plataforma sería capaz de satisfacer los objetivos previstos.

El primer objetivo que se planteó fue el de conseguir un sistema de cámara y movimiento del personaje principal, ya que al ser la parte relacionada con lo que el usuario podrá controlar del videojuego, era necesario realizarla cuanto antes. Todo ésto implicaba también la gestión de animaciones y ajuste de parámetros para obtener el resultado esperado.

Otro objetivo sería la recreación de corte de objetos en un entorno 3D. Para ello se necesitó investigar acerca de las posibiliades y el funcionamiento de las geometrías en \textit{Unity}, y la posibilidad de usar frameworks que ayudaran a desarrollar ésto.

La creación de mapas y sus elementos de forma procedural se planteó y se decidió como forma de crear los mapas ya que al no contar con las herramientas ni conocimientos necesarios para realizar el diseño de los entornos del videojuego, finalmente se optó por la generación automática y aleatoria de los mismos. La forma en la que han planteado los mapas ha sido a modo de \textit{Mazmorra} clásica que consiste en un conjunto de habitaciones unidas por pasillos en los que hay enemigos que derrotar y de donde no se puede salir sin conseguir el objetivo propuesto.

Los enemigos y su comportamiento en el entorno serán otros objetivos a realizar dentro del proyecto. Además, para recompensar al jugador, la eliminación de enemigos conllevará la obtención de mejoras para el jugador y puntos, para motivar al jugador para eliminar más enemigos. 

Otro de los objetivos específicos es añadir efectos visuales y sonoros dentro del videojuego, ya que con ellos el videojuego obtiene más calidad y por tanto será más entretenido.

La implementación de un cauce gráfico es otro de los objetivos principales. Para poder conseguir éste objetivo será necesaria la investigación acerca de cómo los elementos lumínicos en \textit{Unity} actúan sobre las caras de las mallas de polígonos que forman los objetos. También requerirá el aprendizaje herramientas específicas dentro de \textit{Unity} para poder lograr este objetivo.

La victoria del jugador para poder abandonar el mapa y completar la partida se conseguirá al recoger un objeto específico que aparecerá aleatoriamente en el mapa. Debido a esto en ocasiones se puede dar que el jugador aparezca en la misma habitación que el objeto mencionado, pero, recogerlo instantáneamente, pese a que finalice la partida no será el objetivo, ya que existirá un sistema de puntos y puntuación máxima a la que el jugador deberá intentar llegar y superar.

Finalmente, para poder conseguir que el jugador pueda navegar entre las distintas pantallas del videojuego, será necesario desarrollar una interfaz de usuario funcional, para que pueda realizar las acciones de empezar partida, abandonarla y ajustar algunos parámetros.


\chapter{Resolución del trabajo}


En este apartado se expondrán tanto los métodos utilizados para la planificación del 
desarrollo del videojuego como la implementación del mismo.  

\section{Planificación y presupuesto}
WIP (AQUI PONER LAS HORAS ESTIMADAS)

\subsection{Planificación inicial}
WIP

\subsection{Presupuesto}
WIP

\section{Análisis y diseño}

A continuación se va a tratar el análisis del proyecto planteado y su respectivo diseño.

\subsection{Especificación de requisitos}

\subsubsection{Requisitos funcionales}

%Lista de requisitos funcionales
\begin{enumerate}
    \item[\textbf{RF-1}] Control del personaje.
    \begin{enumerate}
        \item[\textbf{RF-1.1}] El personaje tendrá un set de movimientos y acciones básicos.
        \begin{enumerate}
            \item[\textbf{RF-1.1.1}] El personaje podrá caminar hacia cualquier dirección para poder desplazarse por el entorno.
            \item[\textbf{RF-1.1.2}] El personaje tendrá una serie de  movimientos de ataque para derrotar a los enemigos que encuentre.
            \item[\textbf{RF-1.1.3}] El personaje tendrá la capacidad de poder cortar elementos del entorno.
        \end{enumerate}
        \item[\textbf{RF-1.2}] El jugador deberá poder controlar los movimientos del personaje.
        \item[\textbf{RF-1.3}] El jugador tiene que poder controlar las acciones del personaje.
        \item[\textbf{RF-1.4}] El videojuego se deberá poder controlar tanto con mando como teclado y ratón.
    \end{enumerate}
    \item[\textbf{RF-2}] Comportamiento del entorno.
    \begin{enumerate}
        \item[\textbf{RF-2.1}] El sistema generará una partida completamente nueva y distinta a la anterior cada vez que se inicie un nuevo juego.
        \begin{enumerate}
            \item[\textbf{RF-2.1.1}] Las mazmorras se generarán de manera aleatoria para evitar que el videojuego se vuelva repetitivo.
            \item[\textbf{RF-2.1.2}] Los enemigos de las mazmorras se generarán en patrones distintos y en salas distintas.
            \item[\textbf{RF-2.1.3}] Las recompensas o mejoras para el jugador aparecerán en situaciones aleatorias.
        \end{enumerate}
        \item[\textbf{RF-2.2}] El sistema generará a los enemigos que serán agentes reactivos.
        \item[\textbf{RF-2.3}] El sistema generará a los enemigos serán de distintos tipos y cada tipo tendrá su propio comportamiento.
        \item[\textbf{RF-2.4}] El sistema generará recompensas y mejoras para el jugador. 
        \begin{enumerate}
            \item[\textbf{RF-2.4.1}] Las recompensas serán de varios tipos.
            \begin{enumerate}
                \item[\textbf{RF-2.4.1.1}] El sistema generará recompensas que mejoren el daño provocado a los enemigos.
                \item[\textbf{RF-2.4.1.2}] El sistema generará recompensas que mejoren la resistencia del jugador.
            \end{enumerate}
        \end{enumerate}
        \item[\textbf{RF-2.5}]El sistema generará mazmorras que tendrán distintos tipos de salas.
        \begin{enumerate}
            \item[\textbf{RF-2.5.1}] El sistema generará salas con enemigos las cuales no se desbloquearán hasta derrotar a todos.
            \item[\textbf{RF-2.5.2}] El sistema generará salas con recompensas. 
            \item[\textbf{RF-2.5.3}] El sistema generará salas con un enemigo tipo jefe que presentará un mayor desafío para el jugador. 
        \end{enumerate}
        \item[\textbf{RF-2.6}] El sistema será capaz de reproducir sonidos acordes a lo sucedido en el videojuego.
    \end{enumerate}
    \item[\textbf{RF-3}] Control sobre el sistema
    \begin{enumerate}
        \item[\textbf{RF-3.1}] El jugador podrá empezar una partida.
        \item[\textbf{RF-3.2}] El jugador podrá guardar una partida.
        \item[\textbf{RF-3.3}] El jugador podrá abandonar una partida.
        \item[\textbf{RF-3.4}] El jugador podrá pausar la partida.
        \item[\textbf{RF-3.5}] El jugador deberá ser capaz de cambiar distintos parámetros del sistema.
        \begin{enumerate}
            \item[\textbf{RF-3.5}]  El jugador deberá ser capaz de ajustar el nivel de sonido.
            \item[\textbf{RF-3.6}]  El jugador deberá ser capaz de ajustar la resolución de pantalla.
        \end{enumerate} 
    \end{enumerate}
\end{enumerate}

\subsubsection{Requisitos no funcionales}


\subsection{Material importado}

Como en este proyecto la intención ha sido desde el principio centrarse en la parte técnica del desarrollo
del videojuego, los diseños y modelos utilizados en su mayoría han sido importados de webs las cuales
ofrecen modelos con licencia \textit{Creative Commons} para uso libre. A continuación se mostraran los modelos utilizados 
y de donde se han obtenido.\\
Para este proyecto serán necesarios bastantes elementos los cuales son:
%Aqui tengo que poner también ademásd del personaje, los muebles, enemigos, armas y decorados que use.

\subsubsection{Personaje principal}
Este modelo ha sido importado desde la web de \textit{Adobe Mixamo} \cite{Mixamo} la cual ofrece una gran cantidad de modelos 3D
y animaciones, la gran mayoría compatibles con \textit{Unity}. Debido a el contexto en el que se quiere situar
el videojuego, el modelo elegido es una especie de androide humanoide futurista. Dado que es algo que no se va a tener en cuenta a la hora de
jugar, el personaje principal no será personalizable, siempre tendrá este aspecto. \\
A continuación imagen del modelo elegido: 

\begin{figure}[htbp]
\centering
\includegraphics[width=8cm, height=8cm]{characterModel.jpg}
\caption{Modelo Alien Soldier}
\end{figure}

Un aspecto bastante interesante de este modelo es que es \textit{rigged}, es decir, que incluye
un esqueleto con el que a la hora de añadir las distintas animaciones al personaje, se va a facilitar bastante dicha tarea. \\

Para animar este modelo se han importado también una serie de animaciones, también de la web
\textit{Mixamo}. Dichas animaciones abarcan movimientos como caminar, correr, saltar y atacar.

\subsubsection{Armas}
En esta sección se mostraran los modelos de armas utilizados por el personaje principal y enemigos del videojuego

La espada del personaje principal, es una \textit{katana} la cual se importó desde la web de 
\textit{Sketchfab} \cite{Sketchfab} que ofrece gran cantidad de Assets y modelos 3D tanto gratis como de pago.
Este modelo seleccionado es gratuito y fue creado por el usuario shor.riot. \\

Se eligió este tipo de arma ya que es bien conocido
que las katanas japonesas tienen fama de ser muy afiladas y por tanto, para un aspecto importante del videojuego 
(el modo Ultra en el que el personaje será capaz de cortar con su katana a los enemigos y elementos que tenga delante) es ideal que sea esta arma, y se
ha escogido este modelo en concreto ya que con la luz neón verde que incluye la textura del modelo,
da la sensación de ser una katana más futurista y así adaptarse mejor al contexto del videojuego. A continuación se muestra el modelo
de dicha katana: 

\begin{figure}[H]
    \centering
    \includegraphics[width=8cm, height=8cm]{phaseKatana.jpg}
    \caption{Modelo 3D Phase Katana}
\end{figure}

\subsubsection{Enemigos}

Los modelos de los enemigos han sido importados desde una página web la cual es la que más elementos ha proporcionado al proyecto, dicha página es \textit{Unity Asset Store} \cite{UnityAssetStore}. Según la propia página de documentación oficial proporcionada por Unity, la \textit{Unity Documentation} \cite{UnityDocumentation} este servicio es una plataforma en línea donde los desarrolladores de videojuegos pueden encontrar y adquirir una amplia variedad de activos, recursos y herramientas para usar en sus proyectos de Unity. Estos activos pueden incluir modelos 3D, texturas, scripts, paquetes de efectos visuales, música, sonidos, plantillas de interfaz de usuario y mucho más.

Unity Asset Store proporciona a los desarrolladores una forma conveniente de expandir y mejorar sus juegos al ofrecer una amplia selección de contenido creado por otros desarrolladores. Los activos se pueden comprar o descargar de forma gratuita, dependiendo de las preferencias del creador del activo. Además, también existe la opción de adquirir paquetes completos que contienen varios activos relacionados, en este caso, este paquete es gratuito e incluye todo lo necesario para la funcionalidad que se quiere hacer.

El paquete de modelos 3D, texturas y animaciones de los enemigos es \textit{SciFi Enemies and Vehicles} del usuario \textit{Popup Asylum}. Dicho paquete fue de gran utilidad y calidad pese a ser gratuito. El modelo 3D usado para el enemigo de tipo \textit{Melee} o ataque a corta distancia, es una especie de escorpión robótico, llamado por el autor como \textit{Warrior}. A continuación imagen de dicho modelo.

\begin{figure}[H]
    \centering
    \includegraphics[width=10cm, height=8cm]{MeleeEnemy.jpg}
    \caption{Modelo 3D enemigo Melee Warrior}
\end{figure}

Las animaciones de este enemigo también están incluidas en el paquete. Las que se han usado han sido animaciones de patrullar, correr, observar y atacar.

\subsubsection{Decoración}

Otro aspecto importante de los videojuegos es el propio mapa y los elementos que lo decoran, esto sirve para poder acercar lo máximo posible al usuario 
el entorno o historia que se quiere transmitir. Para ello hay que diferenciar entre los distintos elementos que conforman el concepto de decoración.

Lo primero de lo que hablaremos será de la \textbf{Skybox} del entorno. De acuerdo con la \textit{Unity Documentation} una skybox es una envoltura alrededor de la escena que muestra cómo se ve el mundo más allá de su geometría. Es decir, es el paisaje de fondo que existe en el videojuego, además de los elementos 3D, es necesario una skybox para conseguir la sensación de estar dentro de un mundo inventado por el creador del videojuego.\\

Para las skybox normalmente se usan  \textit{Cubemap} \cite{Cubemaps} las cuales son una representación especial de una textura en forma de cubo en tres dimensiones. Consiste en seis texturas 2D separadas que se unen para formar un cubo completo. Cada una de las seis caras del cubo representa una vista diferente del entorno y además cada una de las caras tienen las mismas dimensiones. A continuación un ejemplo de dicho tipo de imágenes.\\

\begin{figure}[H]
    \centering
    \includegraphics[width=8cm, height=8cm]{cubemapExample.png}
    \caption{Ejemplo de Cubemap usado para una Skybox}
\end{figure}

Sin embargo en mi caso se ha utilizado otra variante también muy usada, la cual se trata de las \textit{Equirectangular Images} \cite{EquirectangularImages}. Se trata una imagen 2D que se envuelve alrededor de una esfera, proporcionando un campo de visión horizontal de 360 grados y un campo de visión vertical de 180 grados. Este tipo de proyección se utiliza comúnmente para imágenes panorámicas y ofrece una forma conveniente de capturar y mostrar una vista de gran angular de un entorno, como por ejemplo el típico mapa de La Tierra.
Una vez explicado este concepto, pasamos a explicar la creación de la Skybox usada en el proyecto, puesto que es una imagen única producida por una inteligencia artifical generativa. Esta herramienta se encuentra en la web y está en desarrollo, se trata de \textit{Blockade Labs} \cite{BlockadeLabs}, una web que ofrece de forma gratuita una IA generativa, la cual, a partir de inputs de texto y opciones seleccionables, es capaz de generar Skyboxes ciñendose a dichos inputs. Pues de esta manera se generó la skybox para el proyecto, a continuación se muestra dicha imagen equirectangular.

\begin{figure}[H]
    \centering
    \includegraphics[width=12cm, height=8cm]{skybox.jpg}
    \caption{Skybox generada por IA y usada en el proyecto}
\end{figure}

Como se puede observar, se eligió un concepto futurista y en entorno nocturno para generar dicha imagen, ya que se ajusta bien a el objetivo y contexto del videojuego.\\

Siguiendo con el decorado de los escenarios, se buscaron elementos que pudieran encajar en la temática. Se decidió buscar en la ya mencionada \textit{Unity Asset Store} y se encontró un paquete con una gran cantidad de modelos de decorado 3D, y que además encajaban con la temática \textit{SciFi}. Este paquete es \textbf{Low Poly Sci Fi Set} del usuario \textbf{Walter Palladino}. De este paquete se han usado 6 elementos de decoración, como son cajas de distinto tamaño y forma, así como de 2 cápsulas de cristal, una rota y otra en buen estado. Estos elementos están escogidos para que además el jugador pueda cortarlos con la mecánica del \textbf{Modo Ultrasónico} ya que a veces estos elementos bloquearán el camino y el jugador deberá abrirse paso cortándolos.
Estos elementos se muestran a continuación.

\begin{figure}[H]
    \centering
    \includegraphics[width=12cm, height=8cm]{caja1.jpg}
    \caption{Modelo 3D caja con una línea en medio color verde}
\end{figure}

\begin{figure}[H]
    \centering
    \includegraphics[width=12cm, height=8cm]{caja2.jpg}
    \caption{Modelo 3D de cubo simple}
\end{figure}

\begin{figure}[H]
    \centering
    \includegraphics[width=12cm, height=8cm]{caja3.jpg}
    \caption{Modelo 3D de caja con 3 líneas de decoración color verde}
\end{figure}

\begin{figure}[H]
    \centering
    \includegraphics[width=12cm, height=8cm]{caja5.jpg}
    \caption{Modelo 3D de caja rectangular con una línea de color verde}
\end{figure}

\begin{figure}[H]
    \centering
    \includegraphics[width=12cm, height=8cm]{capsulas.jpg}
    \caption{Modelos 3D de capsulas de cristal, una rota y otra en buen estado}
\end{figure}

Se importaron de otro paquete otros 2 elementos más de decoración, ampliando hasta 8, el total de elementos que conforman el decorado del videojuego. El paquete se llama \textbf{LowPoly Server Room Props} del usuario \textbf{iPoly3D}.

A continuación imagenes de los 2 modelos seleccionados.

\begin{figure}[H]
    \centering
    \includegraphics[width=12cm, height=8cm]{caja4.jpg}
    \caption{Modelo 3D de caja rectangular con adornos en la parte delantera}
\end{figure}

\begin{figure}[H]
    \centering
    \includegraphics[width=12cm, height=8cm]{panel.jpg}
    \caption{Modelo 3D de panel de mandos}
\end{figure}

Por último, también se ha añadido decorado para el propio personaje, se trata de otro modelo de katana distinto al mencionado anteriormente, y se va a incluir en esta sección ya que su uso va a ser puramente estético, no va a ser un arma funcional dentro del videojuego. Este modelo también ha sido descargado e importado desde \textit{Unity Assets Store}, se trata de un modelo de katana 3D junto con la funda de la misma. El paquete se llama \textbf{Free Katana and Scabbard} de el usuario \textbf{Hideout Studio}.\\

Dicha funda y katana se han añadido en la cintura del personaje, el proceso seguido para conseguir que la katana esté ceñida a la cintura del personaje será explicado más adelante en la sección de Implementacion. A continuación el modelo mencionado:

\begin{figure}[H]
    \centering
    \includegraphics[width=12cm, height=8cm]{modeloKatana2.jpg}
    \caption{Katana y funda 3D importada para mejorar el aspecto del personaje principal}
\end{figure}

\subsubsection{Coleccionables}

Para los objetos coleccionables que el jugador podrá recoger por el escenario, se han importado también de \textit{Unity Asset Store}. El paquete es \textbf{Ten Power-Ups} del usuario \textbf{TeKniKo}, y dicho paquete incluye numerosos iconos y modelos prefabricados de coleccionables, además de un Script que produce el efecto de flotar en el aire. Los coleccionables usados de dicho paquete han sido, una estrella la cual al recogerla se acaba la partida y el jugador gana, unas flechas verdes apuntando hacia arriba indicando mejora de daño, y un icono de una cruz roja para curar la salud del jugador.

\begin{figure}[H]
    \centering
    \includegraphics[width=12cm, height=8cm]{powerUps.jpg}
    \caption{Coleccionables usados en el proyecto, estrella, flechas verdes y cruz roja}
\end{figure}


\subsubsection{Sonidos}

Los sonidos usados en el proyecto, han sido obtenidos de la plataforma \textit{YouTube} \cite{YouTube}. Todos los sonidos y la música obtenida es de uso libre y no tienen licencias de \textit{Copyright}.\\

De sonidos de efectos especiales se han importado los siguientes: 
\begin{itemize}
    \item[\textbf{Correr}] Este sonido se ha obtenido del vídeo llamado \textbf{RUN / RUNNING SOUND EFFECT | FOOTSTEPS SOUND [High Quality]} del usuario \textbf{LISTEN}, y este sonido se ha usado para el efecto especial del personaje cuando realiza la acción de correr.
    \item[\textbf{Corte de Katana}] Sonido conseguido del vídeo \textbf{Katana Swing Cut - Sound Effect for editing} del usuario textbf{Sound library}, usado cuando el jugador realiza un corte con la katana a un objeto.
    \item[\textbf{Espadazo}] El vídeo de donde se ha sacado se llama \textbf{sword slash (sound effects) || mani creation ||} del usuario \textbf{Become a better you}. Para este sonido he tenido que recortarlo ya que en la misma pista de audio había muchos más efectos.
    \item[\textbf{Cámara lenta}] Sonido usado para indicar al jugador, además de forma visual, de forma auditiva que ha entrado en \textit{Modo Ultrasónico}. El vídeo de donde se ha obtenido el efecto es \textbf{Slow Motion Sound Effect} del usuario \textbf{SFX Sounds}. Este efecto también ha habido que recortarlo ya que vienen juntos en la misma pista varios efectos.
    \item[\textbf{Power Up}] Efecto de recoger un power up o coleccionable. Vídeo \textbf{Power-Up - Sound Effect (HD)} del usuario \textbf{House Of Sound Effects}.
    \item[\textbf{Música partida}] Esta canción suena de fondo mientras transcurre la partida, como ya se ha mencionado antes es música sin licencias de \textit{Copyright}. \textbf{Sci Fi Cyberpunk - VHS [Synthwave/Electro]} del usuario \textbf{The Neon World}.
    \item[\textbf{Música Menú }] Canción que suena durante el tiempo que el jugador esté en el menú principal, se ha obtenido del vídeo \textbf{Synthwave Game Boy by Infraction [No Copyright Music] / Cassette} y el usuario que lo proporciona es \textbf{Infraction - No Copyright Music}
    \item[\textbf{Música victoria}] Esta música sonará cuando el jugador recoja el coleccionable con el que se acaba la partida, se ha descargado desde el vídeo \textbf{Edge of Tomorrow - Synthwave - Royalty Free Music} del usuario \textbf{TeknoAXE's Royalty Free Music}.  
\end{itemize}

\subsection{Descripción detallada del proyecto }

A la hora de hablar del desarrollo del proyecto, es posible realizar una distinción entre
 distintas partes del proyecto y pasar a su explicación de forma separada, ya que cada una de estas 
 partes abarca herramientas y enfoques distintos. Más adelante en la sección de implementación se abarcarán los detalles técnicos de cada sección.

\subsubsection{Aspectos generales del proyecto}

El proyecto ha sido desarrollado en la versión 2021.3.2f1 \cite{UnityLTS} del editor de Unity, que además es una versión \textit{Long Time Support}, por lo que este ha sido el motivo de usar esta versión.\\

Para el estilo visual que se ha implementado en el proyecto, ha sido necesario utilizar el \textit{Universal Render Pipeline} \cite{URP}. Un \textit{Render Pipeline} \cite{RenderPipeline} o tubería de renderizado, es un conjunto de etapas o procesos que se utilizan en los motores gráficos para convertir datos de geometría y texturas en imágenes visuales finales en tiempo real. En Unity existen 3 render pipelines distintas, prefabricadas, además de que se da la posibilidad al usuario de crear su propia render pipeline si así se prefiere.

En Unity, existen dos sistemas principales de renderizado: el sistema de renderizado incorporado (Built-in Render) y el sistema de renderizado universal (Universal Render Pipeline).

El \textbf{Built-in Render} es el predeterminado en Unity. Proporciona una amplia compatibilidad con diferentes dispositivos y plataformas. Permite utilizar características avanzadas como reflejos en tiempo real, sombras en tiempo real y efectos de postprocesamiento. Sin embargo, puede tener un rendimiento inferior en comparación con los sistemas más optimizados.

El sistema \textbf{URP} (Universal Render Pipeline), está diseñado para ofrecer un rendimiento eficiente en dispositivos móviles y de gama baja. Utiliza un enfoque de sombreado simplificado y un conjunto de características restringido en comparación con el sistema Built-in. A pesar de esto, el URP permite utilizar shaders personalizados, lo cual se ha realizado en este proyecto, efectos de postprocesamiento y efectos visuales de alta calidad mediante su propio sistema de efectos. También incluye herramientas adicionales de optimización, como el culling (desaparición) de objetos, para mejorar el rendimiento. Como ya se ha mencionado, este es el sistema que se ha usado en este proyeto.

Además de estos sistemas principales, Unity también ofrece el \textbf{High Definition Render Pipeline} (HDRP) como una opción avanzada para obtener gráficos de alta calidad en dispositivos de gama alta. El HDRP proporciona características de renderizado fotorrealistas, iluminación global avanzada, sombras de alta calidad y efectos visuales realistas. Lógicamente, esta render pipeline no era conveniente para este proyecto por lo que ni se valoró su uso.

Finalmente, debido a las facilidades y la alta capacidad de personalización y rendimiento que ofrece el URP, se valoró como el más adecuado para el proyecto, sobre todo en el aspecto de diseño de un estilo visual diferenciado.

\subsubsection{Cámara del personaje principal} 

Quizás este es uno de los aspectos más importantes de los videojuegos, la \textbf{cámara} es algo
que puede marcar la diferencia entre un videojuego y otro y por supuesto influye en aspectos como
el tipo de videojuego que se va a realizar o la experiencia del jugador/usuario.\\

Antes de hablar de cómo se ha desarrollado la cámara, se va a explicar muy brevemente los distintos tipos de cámara más usados en los videojuegos a día de hoy: 
\begin{enumerate}
    \item[\textbf{1º persona}] Este tipo de cámara es de los más usados actualmente, consiste en recrear con la cámara 
    que el jugador está viendo lo mismo que el personaje que controla dentro del mismo. Este tipo de cámara sobre todo se plantea en videojuegos realizados en entornos 3D .Esta cámara se suele usar sobretodo en 
    videojuegos tipo \textit{shooter} o de miedo, para conseguir esta inmmersion para el jugador. A continuación un ejemplo de dicho tipo de cámara:
    \begin{figure}[H]
        \centering
        \includegraphics[width=10cm, height=5cm]{doomEjemplo.jpg}
        \caption{Videojuego Doom}
    \end{figure}
    \item[\textbf{3º persona}] Este tipo de cámara pretende dar la sensación al jugador de que está presenciando
    en forma de espectador lo que le ocurre al personaje del videojuego, consiguiendo de esa manera que el usuario
    al jugar lo que está haciendo es interferir en la historia o acciones del personaje que controla. Ejemplo a continuación:
    \begin{figure}[H]
        \includegraphics[width=10cm, height=5cm]{BOTWejemplo.jpg}
        \caption{Videojuego The Legend Of Zelda: Breath of the Wild}
    \end{figure}
    \item[\textbf{Isométrica}] Este tipo de cámara, como su nombre indica, se situa en perspectiva 
    Isométrica con respecto a la escena de forma que se ve como si el usuario estuviera situado en el cielo de la escena presenciandola.
    Este tipo de cámara puede ser parecida a la de 3º persona pero con ella se pueden realizar videojuegos totalmente distintos como por ejemplo videojuegos de estrategia o de construcción de ciudades.
    Ejemplo : 
    \begin{figure}[H]
        \centering
        \includegraphics[width=10cm, height=5cm]{AOE2ejemplo.jpg}
        \caption{Videojuego Age Of Empires II}
    \end{figure}
\end{enumerate}

En el proyecto, se va a implementar una cámara en \textbf{tercera persona}. Una vez vistos los principales tipos de cámara que se usan a día de hoy en los videojuegos, proseguiremos con la explicación
del desarrollo de la cámara de el videojuego que se está tratando.

Para la cámara se ha decidido importar un paquete de \textit{Unity} llamado \textit{Cinemachine} \cite{UnityCinemachine} %Referenciar
el cual proporciona una serie de herramientas para facilitar la creación, la lógica y los parámetros de la cámara 
que se va a usar.
\textit{Cinemachine} facilita el uso de la cámara en \textit{Unity} en comparación con la cámara básica de \textit{Unity}, ya que proporciona una forma más intuitiva y fácil de crear y gestionar la cámara. También permite la creación de efectos de cámara avanzados sin necesidad de escribir código, lo que ahorra tiempo y esfuerzo en el desarrollo del juego o aplicación.

Entre las funcionalidades más destacadas de \textit{Cinemachine} se encuentran:

\begin{enumerate}
\item Seguimiento de objetos: \textit{Cinemachine} permite configurar la cámara para seguir automáticamente un objeto determinado, como un personaje, un vehículo, etc. Además, es posible definir el tipo de seguimiento que se desea (por ejemplo, seguir al objeto en todo momento, o solo cuando se mueve) y ajustar la velocidad y otros parámetros.

\item Composición de cámaras: \textit{Cinemachine} permite crear composiciones de cámaras complejas, que pueden incluir varias cámaras configuradas de diferentes maneras. Esto permite crear efectos interesantes, como transiciones entre cámaras o cambios de perspectiva.

\item Efectos de cámara: \textit{Cinemachine} incluye varios efectos de cámara preconfigurados, como la profundidad de campo, la corrección de color, la desenfoque de movimiento, etc. Estos efectos pueden aplicarse fácilmente a la cámara y ajustarse según las necesidades del juego o aplicación.

\item Curvas de animación: \textit{Cinemachine} permite crear curvas de animación para la cámara y otros elementos del juego, lo que permite crear movimientos suaves y naturales. Además, estas curvas pueden ser editadas de manera visual, lo que facilita su ajuste.

\item Integración con otros sistemas: \textit{Cinemachine} se integra bien con otros sistemas de \textit{Unity}, como el sistema de animación o el sistema de física. Esto permite crear efectos más realistas y dinámicos, como movimientos de cámara que se ajustan automáticamente a la física del juego.

\end{enumerate}

En general, \textit{Cinemachine} es una herramienta muy útil para cualquier desarrollador que quiera crear un sistema de cámara avanzado y dinámico para su juego o aplicación. Ofrece una interfaz gráfica intuitiva, una amplia variedad de funcionalidades y efectos de cámara, y se integra bien con otros sistemas de \textit{Unity}.

De acuerdo pues una vez repasados los aspectos principales de \textit{Cinemachine}, la configuración que se ha hecho en el proyecto ha sido, 
seleccionar un tipo de cámara llamada \textit{FreeLook camera} \cite{CinemachineFreelook}. Esta cámara permite 
\begin{itemize}    
    \item \textbf{Control de tres ejes:} La cámara FreeLook permite controlar la posición y rotación de la cámara en tres ejes: horizontal, vertical y de profundidad. Esto permite crear movimientos de cámara complejos y precisos.
   
    \item \textbf{Modo de seguimiento suave:} La cámara FreeLook puede seguir objetos en movimiento con un modo de seguimiento suave que evita movimientos bruscos y mejora la sensación de realismo.

    \item \textbf{Zonas de enfoque:} Es posible definir zonas de enfoque que indican a la cámara qué objetos o áreas deben mantenerse en foco en todo momento. Esto es especialmente útil en juegos de acción o deportes donde los objetos en movimiento pueden desaparecer de la vista rápidamente.

    \item \textbf{Configuración de prioridades:} La cámara FreeLook permite establecer prioridades entre diferentes objetivos de seguimiento. Esto significa que se pueden definir qué objetos tienen más importancia en la escena y la cámara se enfocará en ellos en caso de conflicto.

    \item \textbf{Distancia de seguimiento ajustable:} Es posible ajustar la distancia de seguimiento de la cámara, lo que permite acercar o alejar la cámara del objeto en movimiento para crear diferentes efectos visuales.

    \item \textbf{Modos de enfoque:} La cámara FreeLook tiene diferentes modos de enfoque que permiten controlar cómo se enfoca la cámara en los objetos de la escena. Por ejemplo, se pueden usar modos de enfoque basados en la distancia o en el ángulo de la cámara.
\end{itemize}

Además de esta cámara se ha configurado otro tipo que incluye el paquete \textit{Cinemachine}, que es \textit{Virtual Camera} \cite{CinemachineVirtualCamera}. Esta cámara se parece a la anterior mencionada \textit{FreeLook} pero tiene algunas diferencias por las cuales se ha elegido dicho tipo para cuando el jugador entre en \textbf{Modo Ultrasónico}. Podemos destacar de las propiedades, de las cuales comparte muchas con la cámara \textit{FreeLook} como puede ser el modo de seguimiento (Follow) o LookAt, además ésta permite crear transiciones suaves de una cámara a otra, evitando cambios bruscos y molestos en el jugador. 

Podemos concluir que la diferencia entre ambas es que la Cinemachine FreeLook camera se especializa en proporcionar un movimiento de cámara más complejo y cinematográfico (movimientos de cámara similares a los de películas de cine). Mientras que la VirtualCamera se centra en el seguimiento y la orientación automática de objetivos además de mayor personalización en el ajuste de la cámara como el ángulo o campo de visión. A continuación imagenes con cada tipo de cámara para compararlas.

\begin{figure}[H]
    \centering
    \includegraphics[width=12cm, height=8cm]{FreeLookCamera.jpg}
    \caption{Ejemplo de FreeLook camera realizando un plano contrapicado al personaje}
\end{figure}

\begin{figure}[H]
    \centering
    \includegraphics[width=12cm, height=8cm]{VirtualCamera.jpg}
    \caption{Ejemplo de Virtual camera configurada para simular que tanto el jugador como el personaje estén mirando al centro de la pantalla}
\end{figure}

El Script en Unity
que controla ésta cámara se llama \textit{ThirdPersonCam.cs}, más adelante se verá en detalle el código de dicho script. 
Una vez explicada la cámara y la herramienta utilizada para su desarrollo se procederá a la explicación del siguiente elemento del proyecto.

\subsubsection{Personaje principal}

A la hora de definir los distintos movimientos del personaje principal se han seleccionado una serie 
de animaciones las cuales están sincronizadas con el desplazamiento del personaje, pero vamos a entrar más en detalle para hablar de como se 
ha planteado y desarrollado. Hay que recalcar que las animaciones son un aspecto clave en el desarrollo de videojuegos, pese a que no es el objetivo de este proyecto, se ha cuidado este aspecto todo lo posible durante el desarrollo del proyecto.\\

Lo primero es que se ha hecho es elegir de la web \textit{Unity Asset Store} animaciones de \textit{samurai} japonés, ya que encajaban perfectamente con el planteamiento y las inspiraciones para realizar el proyecto.

 Antes de seguir vamos a definir de forma breve algunos conceptos como el de animación y explicaremos las herramientas con las que se ha realizado este apartado del proyecto. La animación es el proceso de crear la ilusión de movimiento a partir de una secuencia de imágenes estáticas. Se logra mostrando una serie de imágenes en rápida sucesión, cada una ligeramente diferente de la anterior, lo que crea la ilusión de movimiento continuo. Pueden ser dibujos, imágenes generadas por computadora o incluso objetos reales que se mueven cuadro a cuadro. 

 La animación \cite{Animacion} tal y como la conocemos hoy día comenzó cuando los animadores de \textbf{Disney}, \textbf{Ollie Johnston} y \textbf{Frank Thomas}, recogieron en su libro \textit{The Illusion of life} los "Doce principios básicos de la animación" \cite{TheIllusionOfLife}. El objetivo de estos principios era intentar crear la ilusión de que los personajes se apegaban a las leyes de la física aunque se abarcaron también temas como el tiempo emocional y atractivo de los personajes. 

 \begin{figure}[H]
    \centering
    \includegraphics[width=8cm, height=8cm]{pelotaRebotando.png}
    \caption{Bola roja rebotando desglosada en 6 fotogramas. Extraido de https://es.wikipedia.org/wiki/Archivo:Animexample3edit.png }
\end{figure}

 Aunque inicialmente se pretendía que estos principios se aplicaran principalmente a la animación tradicional o animación dibujada a mano, siguen siendo de gran relevancia en la actualidad, especialmente en el contexto de la animación por ordenador que prevalece hoy en día.

Una vez repasado el concepto de animación, ahora se da paso a cómo se puede trabajar con las animaciones en Unity. Recalcar que ninguna de las animaciones usadas ha sido creada de cero para este proyecto, todas han sido creadas por otros usuarios. Lo que sí se ha hecho ha sido modificar la velocidas o algunos fotogramas de determinadas animaciones. 

Para aplicar animaciones en Unity es necesario usar el elemento ''Animator'' \cite{AnimatorUnity} el cual es una herramienta que permite controlar y programar animaciones en un objeto 3D. Funciona mediante la creación de ''animators controllers'', que son conjuntos de reglas y estados que controlan la reproducción de animaciones.

\begin{figure}[H]
    \centering
    \includegraphics[width=12cm, height=8cm]{AnimatorControllerExample.png}
    \caption{Ejemplo de Animator formando un esquema de animator controllers junto a las reglas que los confroman. Imagen extraida de https://docs.unity3d.com/es/530/Manual/class-AnimatorController.html }
\end{figure}

Primero hablaremos de las animaciones del personaje principal, dicho personaje es de tipo ''rigged''. Cuando se habla de un modelo rigged en el contexto de elementos 3D, se refiere a un modelo tridimensional que ha sido equipado con un esqueleto virtual, también conocido como rig. El rig es una estructura interna compuesta por huesos, articulaciones y controladores que simula el sistema musculoesquelético de un personaje o modelo.

Para que haya animaciones distintas y concurrentes en distintas partes del cuerpo, se puede lograr mediante la creación de ''mecanim animations'' \cite{MecanimAnimation}. Este proceso implica la creación de varios animators controllers que controlan diferentes partes del modelo, y luego combinarlos para crear una animación completa y coherente. En el caso del personaje principal se decidió separar el modelo el tren superior e inferior del cuerpo. A continuación imagen de cada máscara creada para indicar que músculos del modelos responderán a las animaciones. 

\begin{figure}[H]
    \centering
    \includegraphics[width=8cm, height=8cm]{UpperBodyMask.jpg}
    \caption{Animator mask correspondiente a el tren superior del cuerpo del personaje, en verde los músculos que se moverán en cada animación}
\end{figure}

\begin{figure}[H]
    \centering
    \includegraphics[width=8cm, height=8cm]{LowerBodyMask.jpg}
    \caption{Animator mask correspondiente a el tren inferior del cuerpo del personaje, en verde los músculos que se moverán en cada animación}
\end{figure}

Para poder acceder a cada una de estas máscaras, dentro del Animator hay que crear tantas ''Layers'' o capas como se deseen, en este caso se crearpn 2 layers, una para cada máscara. En cada layer se presenta un diagrama de estados vacío, de forma que se pueden personalizar cada una de las layers al punto que se desee. En el apartado de la implementación se detallarán los diagramas de estados así como la lógica de transición entre ellos.

El movimiento en sí del personaje está creado vía Script y responde a los inputs del jugador a través de teclas o botones de un controlador.

El personaje principal tiene asociado unos parámetros de Salud y de Daño los cuales se gestionan a través de Scripts según se den las condiciones que los activen o no. Por ejemplo al jugador se le bajará la salud cuando reciba un golpe por parte de un enemigo. Todo esto será posible gracias a los \textit{Colliders} que ofrece Unity, con los cuales se puede gestionar cómo interactúan de forma física los distintos elementos de la escena. Así como se ha dotado al personaje de colliders, a los demás elementos también, ya que sin ellos no se podrá conseguir el comportamiento esperado.

Por último, el personaje también interactuará con una serie de elementos coleccionables los cuales, al colisionar con ellos afectarán a los parámetros de Salud y daño del jugador, incrementándolos. Además de un elemento coleccionable encargado de que cuando el juagdor lo recoja, en ese momento se de por finalizada la partida.

\subsection{Enemigo}

El enemigo que el personaje debe derrotar durante el juego tiene una serie de animaciones asociadas y un comportamiento guiado por una máquina de estados sencilla, la cual contempla 4 estados distintos que son, patrulla, vigilancia, aproximación al jugador y ataque al jugador. La lógica de transición entre animaciones o comportamientos se trata a través de parámetros ''Trigger''. 

Las animaciones de estos enemigos no son de tipo rigged ya que los modelos no son humanoides y por tanto no se tratan igual. Además los enemigos tienen asociados unos parámetros de vida y de daño, los cuales se ven afectados por las acciones del jugador.

\subsubsection{Modo Ultrasónico}

El Modo Ultrasónico consiste en un modo al cual podrá acceder el jugador al mantener pulsada Cierta tecla o botón, y durante el cual, el personaje será capaz de cortar los elementos estáticos del entorno, exceptuando enemicos, paredes y suelo. 

Se ha logrado que solo se corte a cierto tipo de elementos del entorno mediante las 'layers' que ofrece Unity, las cuales se pueden crear para indicar a el editor el layer de cada elemento. Hay muchas layers default, pero se ha creado la layer \textit{Enemy} para esto, y se les ha asociado a cada uno de los elementos cortables esta layer. De esta manera, cuando el script con la lógica de corte interactúa con los colliders de elementos en la capa Enemy, corta dichos elementos.

Para limitar la distancia y la orientación de los cortes, se proporciona al jugador durante este modo, de un plano el cual sólo verá cuando esté activado el modo Ultrasónico, y el cual podrá girar para ajustar los cortes como guste el usuario. 

Durante este modo además se han añadido \textbf{efectos especiales} como son el \textit{Postprocesamiento} y la cámara lenta. Para poder conseguir el efecto de postprocesamiento era necesario aplicar el renderizador URP anteriormente mencionado. Los efectos aplicados son los motrados en la imagen siguiente.

\begin{figure}[H]
    \centering
    \includegraphics[width=8cm, height=12cm]{PostProcessingEffectConfig.jpg}
    \caption{Parámetros de postprocesamiento usados para efectos visuales durante el modo Ultrasónico}
\end{figure}

Con estos efectos visuales se pretende dar la sensación al jugador de que ha entrado en un modo de comportamiento distinto al normal y de esa forma indicar sin que haya texto que puede ser capaz de realizar la acción de corte.

Para conseguir el efecto de cámara lenta, se realiza mediante Script con la propiedad de Unity \textit{Time} \cite{Time}. Time es una clase incorporada que proporciona información sobre el tiempo transcurrido en el juego. Permite a los desarrolladores acceder y controlar el tiempo en su juego, lo que resulta útil para la animación, la física, las transiciones y otros aspectos relacionados con el tiempo.

Algunas de las propiedades y métodos más utilizados de la clase "Time" son:
 
\begin{itemize}    
    \item \textbf{Time.deltaTime} Proporciona la duración en segundos del fotograma anterior. Se utiliza para crear movimientos suaves y consistentes, ya que compensa las diferencias de rendimiento en diferentes plataformas.
   
    \item \textbf{Time.fixedDeltaTime}  Es similar a Time.deltaTime, pero se utiliza en el contexto de las actualizaciones físicas y es constante en cada fotograma.

    \item \textbf{Time.timeScale}  Permite ajustar la velocidad del tiempo en el juego. Un valor de 1.0 significa tiempo normal, mientras que un valor menor ralentiza el tiempo y un valor mayor lo acelera.

    \item \textbf{Time.time} Representa el tiempo transcurrido en segundos desde que se inició el juego. Se utiliza para realizar cálculos basados en el tiempo, como animaciones o temporizadores.
\end{itemize}

Por tanto, la propiedad usada para hacer el efecto de cámara lenta es usar \textbf{Time.timeScale} y asignarle un valor menor que 1.

\subsubsection{Generación procedural del mapa y sus elementos}

Como ya se ha mencionado anteriormente en la introducción del proyecto, este videojuego va a tener una particularidad que no suelen tener otros, el cual es la generación procedural de los escenarios y de los elementos que los conforman. Se ha decidido plantear los escenarios a modo de mazmorra con diferentes habitaciones y pasillos que unen unas habitaciones con otras. Las habitaciones, serán rectangulares y los pasillos también.

Para poder realizar este apartado del proyecto ha sido necesario buscar por la web maneras de plantear la generación del mapa. Finalmente se encontraron una serie de vídeos a modo de tutorial de un usuario en los que explicaba con detalle un algoritmo en el cual se iba a basar a la hora de realizar la implementación. Estos vídeos se llaman \textbf{Procedural dungeon in Unity 3D Tutorial} del usuario \textbf{Sunny Valley Studio}. El algoritmo se explicará unas secciones más adelante, por ahora sólo trataremos este especto de forma generalizada.

Una vez se generan las distintas habitaciones y son unidas por los pasillos, tenemos como resultado una mazmorra la cual siempre que se inicie una partida cambiará. Además los elementos de decoración y los enemigos serán generados en puntos aleatorios dentro de cada habitación. Para calcular esto, como se tienen las esquinas de cada habitación, es posible calcular puntos aleatorios dentro de ellas, procurando que no aparezcan elementos atravesados unos con otros. Esto se ha logrado a los ya mencionados \textit{Colliders}. A la hora de instanciar estos objetos se genera un punto aleatorio dentro de la habitación por la que vaya el algoritmo y se llama a una función llamada \textit{IsValidPosition} la cual realiza lo siguiente: 

\begin{itemize}    
    \item Comprueba que en la posición dada no hay ningún elemento ya emplazado. Esto se realiza mediante una propiedad de la clase \textit{Physics} la cual es generar dado un punto (el que es candidato) una esfera de radio personalizable (en este caso 1) la cual es un \textit{Collider} y comprueba que con esa esfera no haya ninguna colisión, ya que esto indicará que el espacio está libre.
    \item Comprobueba además de lo anterior que el elemento esté a una distancia razonable de otros elementos para evitar que haya muchos elementos concentrados en un lugar de la habitación y el resto esté muy vacío. Esto se consigue realizando lo mismo que antes pero con una esfera de mayor tamaño y realizando las mismas comprobaciones.
\end{itemize}

Por supuesto, las colisiones con el suelo no se tienen en cuenta ya que si no, la esfera generada simpre devolvería como punto inválido. A continuación un apoyo visual de como funciona la \textit{OverlapSphere}.

\begin{figure}[H]
    \centering
    \includegraphics[width=10cm, height=10cm]{OverlapSphere.jpg}
    \caption{Gráfico donde se muestra el comportamiento de una OverlapSphere}
\end{figure}

Para los enemigos se ha aplicado exactamente esta misma lógica. Más delante, en la sección de implementación se entrará en detalle de los distintos componentes y la propia implementación de este algoritmo.

\subsection{Efectos especiales de partículas}

En los videojuegos es muy común y un aspecto importante apoyar todos los diseños de los escenarios y personajes con efectos especiales visuales, con el objetivo de mejorar la experiencia del usuario jugador. Esto, la mayoría de las ocasiones, en Unity se consigue mediante el \textit{Particle System} \cite{ParticleSystem} que el editor tiene incluido.

Los efectos que se han creado en este proyecto han sido tres, efectos de chispas cuando la espada del jugador impacta con un enemigo o realiza un corte, efecto de electricidad alrededor de la espada del jugador y efecto de velocidad cuando el personaje se mueve. Todos estos efectos se consiguen mediante la regulación de ciertos parámetros que más adelante se indicarán. Además para los efectos de chispas y de velocidad se ha diseñado una pequeña lógica para regular cuando deben mostarse dichos efectos.

Para conseguir un mayor efecto visual además, a las chispas se las ha dotado de un comportamiento parecido al de la realidad, ya que cuentan con físicas y rebotan contra elementos del entorno, así como la reducción del tamaño y de la velocidad conforme pasa el tiempo así como el cambio del color de las chispas, para simular el ''ciclo de vida'' de una de ellas. Los demás efectos no cuentan con estas propiedades ya que se ha decidido que no es necesario.

Con el conjunto de estos efectos especiales se busca dar feedback al usuario de forma visual e integrada en el propio entorno del videojuego sin necesidad de utilizar cuadros de texto o indicadores, mejorando la experiencia e inmersión del jugador.

\subsubsection{ToonShading}

Dentro de los videojuegos, además de las mecánicas y diseño de mapas, es muy importante tener un estilo visual o gráfico propio de ese videojuego. Los gráficos de los videojuegos hoy día son algo que se tienen muy en cuenta para poder diferenciar claramente unos de otros. En muchas ocasiones los juegos comparten aspectos mecánicos con otros, pero la gente no los asocia como iguales al tener estilos visuales diferentes. 

Existen varios estilos visuales los cuales son muy populares como el estilo realista, pero en este proyecto se ha optado por el estilo visual ''Toon'' el cual da la propiedad a los elementos del entorno de parecer estar sacados directamente de una película de animación o de un cómic. Se ha optado por este estilo ya que además de ser uno de los más gustados por el público general, porque implicaba un estudio de los \textit{Shaders} en los videojuegos y de las técnicas que hay que realizar para lograr estos efectos en las texturas de los objetos del videojuego.

Hay que distinguir entre \textit{Shader}, \textit{Material} y \textit{Textura} para comprender exactamente qué es lo que se ha realizado. De acuerdo con la documentación de Unity:

\begin{itemize}    
    \item \textbf{Shader} Son Scripts que contienen los cálculos matemáticos y algoritmos para calcular el color de cada pixel renderizado, basándose en el input de iluminación y la configuración del Material, algo clave para poder conseguir el efecto ''Toon''.
   
    \item \textbf{Material} Un material es una configuración que se aplica a los objetos 3D para determinar su apariencia visual. Un material define cómo interactúa la luz con la superficie de un objeto y cómo se reflejan, refractan o absorben los diferentes componentes de la luz.

    \item \textbf{Textura}  Las texturas son imágenes que se aplican a la superficie del objeto para darle detalles visuales. Pueden contener información de color, brillo, rugosidad, transparencia, entre otros. Unity admite diferentes tipos de texturas, como texturas albedo (color base), texturas de normales (para simular detalles de superficie), texturas de especularidad (para reflejos especulares) y texturas de emisión (para generar luz desde la superficie del objeto).
\end{itemize}

Para conseguir el efecto deseado, por tanto era necesario actuar sobre los shaders, y para ello además de usar el URP, hay que configurar un \textit{Shader Graph} \cite{ShaderGraph} de Unity. Un Shader Graph en Unity es una herramienta visual que permite crear shaders de manera interactiva y sin necesidad de programación.

El Shader Graph proporciona una interfaz gráfica de nodos, donde los nodos representan diferentes operaciones y efectos visuales, y las conexiones entre ellos definen cómo se combinan y se aplican esos efectos. Al conectar nodos en el Shader Graph, se crea una representación visual del shader que define cómo se procesa la luz y los materiales en un objeto. A continuación ejemplo de shader graph:

\begin{figure}[H]
    \centering
    \includegraphics[width=10cm, height=8cm]{ejemploShaderGraph.jpg}
    \caption{Nodos usados en el shader graph del proyecto}
\end{figure}

Los nodos son los distintos rectángulos que se pueden observar en la figura anterior, los cuales se van uniendo las salidas de unos a las entradas de otros para recrear el efecto esperado de dibujo animado.

\subsection{Bocetos del diseño del proyecto}

En este apartado se mostrarán los distintos bocetos que se generaron acerca del diseño del videojuego y los modelos. Además, hay algunos bocetos de la interfaz de usuario y la navegación entre ellos, la cual está marcada con un número que indica al boceto al que se dirige en el caso de accionar o pulsar ese icono.

\begin{figure}[H]
    \centering
    \includegraphics[width=12cm, height=8cm]{Bocetoo1.png}
    \caption{Boceto 1}
\end{figure}

\begin{figure}[H]
    \centering
    \includegraphics[width=12cm, height=8cm]{Bocetoo2.png}
    \caption{Boceto 2}
\end{figure}

\begin{figure}[H]
    \centering
    \includegraphics[width=12cm, height=8cm]{Bocetoo3.png}
    \caption{Boceto 3}
\end{figure}

\begin{figure}[H]
    \centering
    \includegraphics[width=12cm, height=8cm]{Bocetoo4.png}
    \caption{Boceto 4}
\end{figure}

\begin{figure}[H]
    \centering
    \includegraphics[width=12cm, height=8cm]{Bocetoo5.png}
    \caption{Boceto 5}
\end{figure}

\begin{figure}[H]
    \centering
    \includegraphics[width=12cm, height=8cm]{Bocetoo6.png}
    \caption{Boceto 6}
\end{figure}

\begin{figure}[H]
    \centering
    \includegraphics[width=12cm, height=8cm]{Bocetoo7.png}
    \caption{Boceto 7}
\end{figure}

\begin{figure}[H]
    \centering
    \includegraphics[width=12cm, height=8cm]{Bocetoo8.png}
    \caption{Boceto 8}
\end{figure}

\subsection{Diagrama de clases}

Para poder realizar este proyecto ha sido necesario la implementación de varias clases las cuales no pertenecen a Unity. Hay una gran cantidad de Scripts en el proyecto (casi 40) pero no todos podrían considerarse clases como tal, ya que algunos solo actúan como scripts. A continuación, se van a dividir por apartados cada clase y se explicara la función de cada uno y posteriormente se expondrá un diagrama de clases donde se podrá visualizar de forma gráfica. Se va a explicar cada script siguiendo más o menos el orden en el que se desarrollaron.

Hay que nombrar primero que la gran mayoria de estas clases van a heredar de la clase \textit{MonoBehaviour}, la cual es una clase base que se utiliza para crear scripts que pueden adjuntarse a objetos en la escena de Unity y controlar su comportamiento. Tiene como métodos imprescindibles los métodos:

\begin{itemize}    
    \item \textbf{Start()} Este método se llama antes de que se actualice el primer fotograma del juego. Se utiliza para inicializar variables y configuraciones adicionales.
   
    \item \textbf{Update()} Este método se llama en cada fotograma del juego. Se utiliza para realizar actualizaciones continuas en el objeto de juego, como movimientos, animaciones o lógica del juego.

    \item \textbf{Awake()}  Este método se llama cuando se instancia el objeto que contiene el script. Se utiliza para inicializar cualquier estado o configuración necesaria antes de que comience el juego.
\end{itemize}

\subsubsection{Cámara y personaje}

Estos Scripts y clases que se explicarán a continuación son aquellos relacionados directamente con el comportamiento del personaje, sus movimientos, acciones y cámara.

\textbf{ThirdPersonCam} es de los ficheros más importates del proyecto, ya que desde él se gestionan gran cantidad de funcionalidades. Desde dicho script se gestiona el movimiento de la cámara del jugador, el cual se realiza mediante los inputs de movimiento de ratón o de Joystick que realiza el usuario. También gestiona la lógica de transición entre modo ''normal'' del jugador y el modo ''Ultrasónico'', además de gestionar el movimiento del plano generado en pantalla para guiar los cortes y la propia gestión de los cortes. Tiene gran cantidad de referencias de los objetos de los que toma propiedades necesarias, por ejemplo para la dirección de la cámara se toman los valores de rotación y posición del objeto del jugador. Además tiene varios atributos de personalización por ejemplo un atributo para asignar un material a los cortes realizados. Este fichero está asignado a un objeto tipo cámara llamado \textit{PlayerCam}.

\textbf{PlayerMovement} como su nombre indica, está encargado de realizar los movimientos del jugador. Tiene gran cantidad de parámetros de ajuste como la fuerza de salto o la velocidad de movimiento, además de parámetros de asignación de teclas y referencias al Animator del personaje, para poder gestionar las animaciones desde él. Este script, va comprobando fotograma a fotograma (método Update()) que input por parte del usuario recibe, y según la tecla pulsada realiza una acción u otra. Algunas de las acciones que realiza es, correr, atacar haciendo hasta un combo de 4 golpes consecutivos y salto. Por supuesto también realiza la lógica de transición de una animación a otra.

\textbf{ultraModeAttacks} es un script auxiliar para ayudar a gestionar el modo Ultrasónico ya que durante este modo algunas de las funcionalidades del jugador quedan deshabilitadas y para evitar problemas, se creó este Script. En este script se gestiona la cámara lenta y los inputs del usuario para que la animación de corte quede lo más precisa posible.

\textbf{Ragdoll} este fichero tiene una función muy específica y que solo puede ocurrir una vez durante la partida, la cual es la ''muerte'' del jugador, durante la cual, el personaje adquiere las propiedades de una muñeca de trapo y se desploma siguiendo las físicas implementadas hasta impactar con el suelo. Pues este fichero se encarga de activar las propiedades que gestionan este comportamiento cuando el personaje se queda sin salud. Para ello, de forma recursiva, activa todos los colliders del personaje y se le despoja del componente Animator.

\textbf{PlayerAttributes} esta clase es de las más importantes ya que gestiona todos los parámetros del jugador, como son la salud y el daño que éste produce. Recibe la información del script más importante llamado \textit{AttributesController}, del que hablaremos más adelante. Este script gestiona cuando el personaje recoje un coleccionable que potencia su daño o su salud, y también cuando el personaje recibe un golpe por parte de un enemigo. Si la salud llega a 0 establece el modo ''Ragdoll''.

\textbf{PowerUpsUI} Script muy sencillo que simplemente se encarga de gestionar el multiplicador del daño en la interfaz de usuario. Este fichero vuelve a leer del script AttributesController.

\textbf{HealthBar} este fichero es muy parecido al anterior y se encarga de gestionar la visualización de la barra de vida del usuario. 

\begin{figure}[H]
    \centering
    \includegraphics[width=13cm, height=12cm]{DiagramaClases1.png}
    \caption{Diagrama de clases correspondiente al personaje y la cámara}
\end{figure}

\subsubsection{Lógica del juego}

A continuación se explicarán los scripts encargados de gestionar la lógica de juego, la cual es, los sonidos, los atributos de enemigos y personaje principal, y la puntuación.

\textbf{AtributesControler} se trata de la clase más importante del proyecto, a pesar de ser más sencilla que otras, ya que casi todas las clases necesitan consultar los atributos de esta clase para realizar acciones determinadas. En esta clase se lleva el seguimiento de parámetros como atributos del jugador y de los enemigos, los multiplicadores de daño, realiza la gestión de la interfaz de usuario, haciendo aparecer o desaparecer pantallas de victoria o derrota y por último, la modificación de parámetros como curación del jugador o aumento del daño producido del mismo. Este fichero esta adjunto a un objeto invisible dentro de la escena, de esta manera es posible que la alteración de los distintos parámetros que posee se realice con el transcurso de la partida. Esta clase además está dotada de varios métodos públicos ya que los demás scripts que necesiten modificar un parámetro, puedan acceder fácilmente a la clase.

\textbf{AudioManager} al igual que la clase anterior, existe también en el proyecto un fichero para gestionar el sonido del videojuego. Este fichero tiene un único atributo el cual es una lista de sonidos. Este fichero también está adjunto a un objeto vacío de la escena. Dicha lista de sonidos ha sido creada a través de la interfaz de Unity a modo de ir añadiendolos uno a uno seleccionando las pistas .mp3 de audio y un nombre. El tipo del array es \textit{Sound} y es una clase creada para tener un tipo con una serie de características que Unity no trae por defecto para los sonidos. El funcionamiento de este fichero es que cuando se invoca a el método \textit{Play(nombre)} busca en la lista de sonidos uno que tenga el nombre proporcionado y crea un componente \textit{AudioSource} el cual emite el sonido que se le adjunte. Este sonido además es 2D por lo que la distancia del personaje no afecta a el volumen del mismo.

\textbf{Sound} es una clase muy sencilla la cual solo contiene atributos como el nombre del sonido, el volument y el pitch, además estos últimos tienen un rango de valores establecido por lo que no será posible asignarles un valor fuera de ese rango (de 0 a 1 por ejemplo).

\textbf{PlayerRespawn} este fichero solo realiza la acción de hacer aparecer el objeto del jugador en el punto que se le pase al método \textit{SpawnPlayer}.

\textbf{ScoreManager} este script realiza la tarea de actualizar visualmente la puntuación del jugador y muestra también el récord actual. Dichos datos son consultados, una vez más, al \textit{AttributesControler}.

\begin{figure}[H]
    \centering
    \includegraphics[width=12cm, height=10cm]{DiagramaManager.jpg}
    \caption{Diagrama de clases correspondiente a los controladores de la lógica del videojuego}
\end{figure}

\subsubsection{Power Ups}

Los siguientes ficheros van a ser los encargados de gestionar los objetos coleccionables del videojuego.

\textbf{HealPowerUp} realiza la gestión de el coleccionable que cura la salud del personaje principal al recogerlo del escenario. Este método se encarga de llamar al controlador de atributos y realiza la acción de sumar en 20 puntos a la vida que tiene el jugador en ese momento. Para ello comprueba que el objeto que colisiona con el es el tag jugador. Además, instancia un efecto especial de recogida el cual da la sensacion de que el jugador absorbe dicho potenciador.

\textbf{DmgPowerUp} hace lo mismo que el script anterior, pero aumentando el multiplicador de daño del jugador, haciendo que éste inflija más daño a los enemigos con cada ataque. Para conseguir esto se vuelve a llamar al controlador de atributos y también cuenta con efectos especiales de recogida.

\textbf{FinishGame} es algo distinto a los demás, ya que cuando es recogido, debe entrar en modo ''partida finalizada''el cual implica, llamar al gestor de sonidos y quitar toda la música y poner la pista de música de victoria. Las acciones de fin del juego se realizan al llamar a la función \textit{EndGame()} del controlador de atributos. 

\begin{figure}[H]
    \centering
    \includegraphics[width=10cm, height=10cm]{DiagramaPowerUps.jpg}
    \caption{Diagrama de clases correspondiente a las clases de los coleccionables}
\end{figure}

\subsubsection{Menús}

Los siguientes scripts son los que gestionan la interacción del usuario con la interfaz gráfica y menús del videojuego.

\textbf{MainMenu} es el encargado de gestionar los 3 botones del menú principal del juego, el que se muestra al iniciarlo. Cuenta con 3 funciones, una para cada acción. Al clickar en el botón de PLAY se carga la escena de juego y de esa forma, comienza el juego. Otra función activa el menú de opciones y la última cierra la aplicación ya que se corresponde con el botón de EXIT.

\textbf{PauseMenu} realiza varias acciones cuando el usuario entra al menú de pausa. Al accionar la tecla ESC se entra en este menú, se hace visible el cursor y se para por completo el tiempo del juego. Si se pulsa en el boton RESUME se quita la vista del menu y se vuelve al tiempo normal. Si se pulsa el boton QUIT, se cierra la aplicación.

\textbf{GameOverScreen} aparece automáticamente cuando el jugador muere, ya que es llamado por el gestor de atributos. Cuenta con 2 botones y 2 funciones. El boton RESTART al ser accionado, recarga la escena actual dando lugar a una partida nueva. El otro botón devuelve al jugador al menú principal. 

\textbf{FinishGameMenu} es muy parecido al script anterior, ya que también aparece automática por orden del controlador de atributos. Éste menú cuenta con un solo botón de salida al menú principal. Como se puede ver, son scripts muy sencillos, sin embargo se ha decidido separar en varios para mejor organización y modulariación.

\begin{figure}[H]
    \centering
    \includegraphics[width=12cm, height=10cm]{DiagramaMenu.jpg}
    \caption{Diagrama de clases correspondiente a los menús}
\end{figure}

\subsubsection{Enemigos}

Ficheros de control de los enemigos, así como el control de la vida y barra de vida de los mismos.

\textbf{MeleeEnemy} realiza varias acciones. Lo primero que hace es establecer los atributos del enemigo antes del primer frame de juego, dichos atributos los obtiene de la clase \textit{AttributesControler}. Durante el juego, si el enemigo tiene más de 0 de salud, entrará en la función \textit{enemyBehaviour} la cual definirá el comportamiento del enemigo. Se trata de una máquina de estados la cual va cambiando la rutina del enemigo según el tiempo que se haya establecido con un cronómetro. Si está a mas de 7 unidades de distancia del jugador, el enemigo podrá entrar de forma aleatoria en uno de los 2 estados pasivos, o bien mantener la posición a modo de vigilancia, o bien andar para patrullar. Sin embargo, si el jugador se encuentra a menos de 7 unidades de distancia, entra en modo combate, el cual tiene otros 2 estados. El primero de ellos es correr en direccion al jugador, este estado no para o bien hasta que se aleje el jugador o bien llegue a menos de 1 unidad de distancia al jugador. Si ocurre lo segundo, el estado cambiará a ataque, realizando la animación de ataque al jugador. Si la vida del enemigo llega a menos de 0, morirá y realizará la animación correspondiente. Al morir, se generará un número aleatorio comprendido entre 0 y 1, si el valor está en el rango 0 a 5 incluido, aparecerá en el lugar del enemigo eliminado un potenciador  de daño. Si sale un valor mayor de 0.5 saldrá una curación. De esta manera se ha dotado de un comportamiento autónomo a los enemigos y se ha logrado un sistema de recompensas para el jugador en forma de obtención de puntos y potenciadores al eliminar enemigos, motivándolo a seguir adelante.

\textbf{Billboard} hace que un objeto siempre mire hacia la cámara . El objeto tiene una referencia a la cámara principal y a su transformación. En el método \textit{Start()}, se obtiene la referencia a la cámara principal y se guarda su transformación, que tiene parámetros de posición y de rotación. Luego, en el método \textit{LateUpdate()}, se actualiza la rotación del objeto para que siempre esté orientado hacia la dirección de la cámara principal. De esta manera conseguimos que la barra de vida situada encima de los enemigos, siempre sea visible y el jugador pueda ver fácilmente la vida restante de sus oponentes.

\begin{figure}[H]
    \centering
    \includegraphics[width=12cm, height=10cm]{DiagramaEnemy.jpg}
    \caption{Diagrama de clases correspondiente a la clase del enemigo}
\end{figure}

\subsubsection{Mapa}

En esta última sección se explicará sin entrar en demasiado detalle los scripts realizados para la generación procedural del mapa. En el apartado de implementación se explicará con más detenimiento.

\textbf{DungeonCreator} es el script principal de lo que corresponde a la generación del mapa o mazmorra ya que es el script que gestiona todos los parámetros de creación de la mazmorra y la que instancia todas las Mallas y objetos que la conforman. Es el único script que va atado a un objeto de la escena, que es invisible, y además es el único método que hereda de la clase \textit{MonoBehaviour}. En este fichero se gestiona también la lógica ya comentada de comprobación cuando se va a instanciar a un objeto, estar seguros de que no va a instanciarse dentro de otro objeto o muy cerca.

\textbf{DungeonGenerator} es un fichero sencillo que es llamado por el anterior para comenzar a realizar el algoritmo de Partición binaria al llamar a su método \textit{CalculateDungeon} el cual devuelve una lista de nodos correspondientes a las habitaciones y a los pasillos que las unen. Todo el proceso del algoritmo se ha dividido en el resto de ficheros.

\textbf{BinarySpacePartitioner}, la primera clase que es llamada en el método de crear los nodos de la mamorra. Se toman los parámetros de ancho y largo de la mazmorra y el número de iteraciones, y establece un nodo central con dichos parámetros, para que a raíz de él se puedan ir haciendo las divisiones del espacio y creando nodos hijos.

\textbf{StructureHelper}. Una vez obtenida la lista de nodos se llama a StructureHelper para poder obtener los nodos ''hoja'' de la lista obtenida, los cuales corresponden con los posibles nodos de las habitaciones. Además dispone de métodos que calculan las esquinas y el punto medio del nodo dado.

\textbf{Node} es una clase creada para poder recrear lo que es un Nodo de un árbol, ya que en Unity no existe dicha clase por defecto. Tiene gran cantidad de parámetros incluyendo referencia al nodo padre, profundidad del nodo e información del mismo, como las coordenadas. Además cuenta con métodos para añadir los nodos hijos a al nodo padre que represente en ese momento. 

\textbf{RoomNode} clase sencilla que hereda de \textit{Node} y cuenta solamente con un constructor y atributos como el largo y ancho de una habitación, ya que esta clase se ha creado para representar ese tipo de nodos. En el constructor se establecen atributos como el nodo padre, las esquinas de dicha habitación y la profundidad del nodo.

\textbf{RoomGenerator} es la clase encargada de crear los objetos \textit{RoomNode} y devolverlos, cada objeto de ese tipo se crea sí y solo sí, dado el array de espacios libres dados, en cada espacio dado cabe una habitación. Si se da este caso se crea el nodo de habitación y se añade a la lista que se va a devolver.

\textbf{CorridorNode}. Al igual que los nodos de habitación, los nodos de los pasillos también heredan de la clase Node. Cuentan con unos parámetros algo distintos a las habitaciones además de muchos métodos para obtener la relación en el espacio de las habitaciones contiguas, es decir, si una habitación está a la derecha, izquierda, arriba o abajo de otra, y según esa información se puede crear el nodo de pasillo o no.

\textbf{CorridorsGenerator} es muy parecida a \textit{roomGenerator} ya que su única funcion es crear los nodos de pasillo uno a uno y añadirlos al array que se devolverá finalmente tras ese proceso.

\textbf{line} define una clase que representa una línea en un espacio bidimensional. La línea tiene una orientación y coordenadas asociadas y un ENUM que define dos posibles orientaciones de la línea, vertical u horizontal. Esta línea es aquella que divide los espacios en 2 en el algoritmo de partición binaria. Es su único pero muy importante uso.

\begin{figure}[H]
    \centering
    \includegraphics[width=15cm, height=15cm]{DiagramaMapa.jpg}
    \caption{Diagrama de clases correspondiente las clases usadas para la genración procedural del mapa}
\end{figure}

% \input{capitulos/02_EspecificacionRequisitos}

% \input{capitulos/03_Planificacion}

% \input{capitulos/04_Analisis}

% \input{capitulos/05_Diseno}

% \input{capitulos/06_Implementacion}

% \input{capitulos/07_Pruebas}

% \input{capitulos/08_Conclusiones}

%\chapter{Conclusiones y Trabajos Futuros}


%\nocite{*}
\bibliographystyle{unsrt} % Specify the bibliography style (unsrt for numerical order)
\bibliography{bibliografia/bibliografia.bib}\addcontentsline{toc}{chapter}{Bibliografía}
%\bibliographystyle{plain}

% \appendix
%\input{apendices/manual_usuario/manual_usuario}
% %\input{apendices/paper/paper}
% \input{glosario/entradas_glosario}
% \addcontentsline{toc}{chapter}{Glosario}
% \printglossary
% \chapter*{}
% \thispagestyle{empty}


\end{document}
