\chapter{Objetivos}
\label{chap:objetivos}

Desde el comienzo, se había tenido en mente en que el objetivo general del proyecto sería un videojuego, pero, debido a la gran cantidad de opciones que implica ésto, no se tenía muy claro cuales iban a ser las características que iba a tener . Sin embargo, tras jugar a algunos títulos diferentes de videojuegos, se decidió realizar una mezcla de  mecánicas vistas en juegos diferentes pero en uno sólo. Además, la curiosidad de cómo es posible recrear en entornos 3D dichas mecánicas vistas, hizo que se decidieran finalmente los distintos objetivos que iba a tener el proyecto. Por lo que el objetivos general pasó a ser la realización de un video con generación procedural de geometría y modificación de ésta, así como la implementación de un cauce gráfico para conseguir el estilo visual deseado.

Para conseguir realizar el objetivo principal, era necesario adquirir una base de conocimientos y experiencia en la herramienta con la que se iba a desarrollar, en éste caso es el motor de videojuegos \textit{Unity} así como el estudio previo de si Unity sería capaz de satisfacer los objetivos previstos.

El primer objetivo que se planteó fue el de conseguir un sistema de cámara y movimiento del personaje principal, ya que al ser la parte relacionada con lo que el usuario podrá controlar del videojuego, era necesario realizarla cuanto antes. Todo ésto implicana también la gestión de animaciones y ajuste de parámetros para obtener el resultado esperado.

Otro objetivo sería la recreación de corte de objetos en un entorno 3D. Para ello se necesitó investigar acerca de las posibiliades y el funcionamiento de las geometrías en Unity, y la posibilidad de usar frameworks que ayudaran a desarrollar ésto.

La creación de mapas y sus elementos de forma procedural se plateó y se decidió como forma de crear los mapas ya que al no contar con las herramientas ni conocimientos necesarios para realizar el diseño de los entornos del videojuego, finalmente se optó por la generación automática y aleatoria de los mismos. La forma en la que han planteado los mapas ha sido a modo de \textit{Mazmorra} clásica que consiste en el conjunto de habitaciones unidas por pasillos en los que hay enemigos que derrotar y no se puede salir de ella sin conseguir el objetivo propuesto.

Los enemigos y su comportamiento en el entorno serán otros objetivos a realizar dentro del proyecto. Además, para recompensar al jugador, la eliminación de enemigos conllevará la obetención de mejoras para el jugador y puntos, para motivar al jugador para eliminar más enemigos. 

Otro de los objetivos específicos es añadir efectos visuales y sonoros dentro del videojuego, ya que con ellos el videojuego obtiene más calidad y por tanto será más entretenido.

La implementación de un cauce gráfico es otro de los objetivos principales. Para poder conseguir éste objetivo será necesaria la investigación acerca de cómo los elementos lumínicos en \textit{Unity} actúan sobre las caras de las mallas de polígonos que forman los objetos. También requerirá el aprendizaje herramientas específicas dentro de \textit{Unity} para poder lograr éste objetivo.

La victoria del jugador para poder abandonar el mapa y completar la partida se conseguirá al recoger un objeto específico que aparecerá aleatoriamente en el mapa. Debido a ésto en ocasiones se puede dar que el jugador aparezca en la misma habitación que el objeto mencionado, pero, recogerlo instantáneamente, pese a que finalice la partida no será el objetivo, ya que existirá un sistema de puntos y puntuación máxima a la que el jugador deberá intentar llegar y superar.

Finalmente, para poder conseguir que el jugador pueda navegar entre las distintas pantallas del videojuego, será necesario desarrollar una interfaz de usuario funcional, para que pueda realizar las acciones de empezar partida, abandonarla y ajustar algunos parámetros.
