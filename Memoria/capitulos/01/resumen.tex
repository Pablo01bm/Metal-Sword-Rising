\chapter{Resumen}
    El presente trabajo de fin de grado, tiene como objetivos el desarrollo de un software gráfico y lúdico, con especial énfasis en la creación de geometrías y su modificación, así como el desarrollo de un cauce gráfico diferenciado.

    El proyecto parte de la dificultad a la hora de desarrollar videojuegos sobre la creación de los escenarios y elementos que componen dicho software, por lo que, mediante la generación procedural ésta tarea se volvería un tanto más sencilla, ya que para una persona sin nociones de diseño de escenarios podría crearlos. Se explicará cómo ha sido posible ésto gracias al algoritmo de Partición Binaria.
    
    Además la complejidad de recrear sucesos físicos de la realidad en un entorno 3D ha llevado a parte del desarrollo de este proyecto a investigar cómo se pueden recrear algunos de esos fenómenos con gráficos 3D. para lograr ésto ha sido necesaria la investigación de físicas y su funcionamiento en entornos 3D así de cómo generar mallas de polígonos de forma automática.  El movimiento y acciones del personaje que controlará el jugador serán algo que recrear dentro del proyecto así de cómo éste interactuará con elementos del entorno gracias a las animaciones, efectos especiales, partículas e interfaz de usuario implementados.

    También, algo que diferencia a los videojuegos entre ellos es su estilo visual, el cual, para poder personalizarlo es necesario realizar un estudio de cómo se comportan las superficies de los elementos 3D con la luz que haya en el entorno así como la modificación dicho comportamiento para favorecer a conseguir el objetivo final.

    A lo largo de éste documento, se expondrá la planificación y el análisis y solución implementada del programa, la cual en ocasiones será gracias al uso de recursos de código abierto, así como el estudio del motor de videojuegos \textit{Unity} en donde se ha desarrollado el videojuego.

    \textbf{Palabras clave} Unity, videojuego, Generación Procedural, Partición binaria, Toon Shading, animación, Software Gráfico, Producto Punto, Cadena Monótona.
    