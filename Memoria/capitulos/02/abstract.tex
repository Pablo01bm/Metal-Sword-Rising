\chapter{Abstract}

The present final degree project aims to develop a graphical and playful software, with special emphasis on the creation and modification of geometries, as well as the development of a differentiated graphical pipeline.

The project starts from the difficulty of developing video games involving the creation of scenarios and elements that compose such software. By using procedural generation, this task becomes somewhat simpler, as even a person without knowledge of scenario design could create them. It will be explained how this has been made possible thanks to the Binary Partition algorithm, which will be used to implement the procedural generation of the different rooms and corridors.

Furthermore, the complexity of recreating real-world physical events in a 3D environment has led a part of the project's development to investigate how some of these phenomena can be recreated using 3D graphics. Achieving this has required research on physics and their functioning in 3D environments, as well as how to generate polygon meshes automatically. The movement and actions of the character controlled by the player will be something to recreate within the project, as well as how it will interact with elements in the environment through implemented animations, special effects, particles, and user interface.

Also, one aspect that differentiates video games is their visual style, which requires studying how 3D elements' surfaces reflect light as a function of its properties and the lighting in with the lighting in the environment, as well as modifying this behavior to achieve the final objective.

Throughout this document, the planning, analysis, and implementation of the program will be presented. In some cases, open-source resources will be used, as well as a study of the \textit{Unity} game engine, which has been used for this video-game development.

\textbf{Keywords} Unity, video game, Procedural Generation, Binary Partition, Toon Shading, animation, Graphic Software, Dot Product, Monotone Chain.